\documentclass[twoside,11pt]{article}
\usepackage[left=1in, right=1in, top=1in, bottom=1in]{geometry}
\usepackage{amsmath}
\usepackage{amssymb}
\usepackage{amsfonts}
\usepackage{mathtools}
\usepackage{amsthm}
\usepackage{fancyhdr}
\usepackage{enumitem}
\usepackage{siunitx}
\usepackage{booktabs}
\usepackage[hidelinks]{hyperref}
\usepackage{sectsty}
\usepackage{mathrsfs} % mathscr
\usepackage{tikz}
\usepackage{pgfplots}
\usepackage{multicol}
\usepackage{listings}
% \usepackage{amsart}
\usepackage{fontspec}
\usepackage{soul}


% allow H option of figure
\usepackage{float}

% math font (libertine)
\usepackage{libertinus-otf}

% braket
\usepackage{braket}

% physics
% \usepackage{physics}

% define latin modern font environment
\newcommand{\lms}{\fontfamily{lmss}\selectfont} % Latin Modern Roman
% \newcommand{\lmss}{\fontfamily{lmss}\selectfont} % Latin Modern Sans
% \newcommand{\lmss}{\fontfamily{lmtt}\selectfont} % Latin Modern Mono

% % change mathcal shape
% \usepackage[mathcal]{eucal}


% define math operators
\newcommand{\FF}{\mathbb{F}}
\newcommand{\RR}{\mathbb{R}}
\newcommand{\NN}{\mathbb{N}}
\newcommand{\ZZ}{\mathbb{Z}}
\newcommand{\QQ}{\mathbb{Q}}
\newcommand{\XX}{\mathbb{Y}}
\newcommand{\CL}{\mathcal{L}}
% \renewcommand{\d}{\mathrm{d}}
\renewcommand*\d{\mathop{}\!\mathrm{d}}
\DeclareMathOperator*{\argmax}{arg\,max}
\DeclareMathOperator*{\argmin}{arg\,min}
\DeclareMathOperator{\im}{im}
\DeclareMathOperator{\id}{id}
\DeclareMathOperator{\erf}{erf}
\renewcommand{\mod}[1]{\ (\mathrm{mod}\ #1)}

% section font style
\sectionfont{\sffamily\Large}
\subsectionfont{\sffamily\normalsize}
\subsubsectionfont{\bf}

% line spreading and break
\hyphenpenalty=5000
\tolerance=20
\setlength{\parindent}{0em}
\setlength\parskip{0.5em}
\allowdisplaybreaks
\linespread{0.9}

% theorem
% latex theorem
% definition style
\theoremstyle{definition}
\newtheorem{theorem}{Theorem}[subsection]
\newtheorem{axiom}{Axiom}[section]
\newtheorem{definition}{Definition}[section]
\newtheorem{example}{Example}[section]
\newtheorem{question}{Question}[section]
\newtheorem{exercise}{Exercise}[section]
\newtheorem*{exercise*}{Exercise}
\newtheorem{lemma}{Lemma}[section]
\newtheorem{proposition}{Proposition}[section]
\newtheorem{corollary}{Corollary}[section]
\newtheorem*{theorem*}{Theorem}
\newtheorem{problem}{Problem}
% remark style
\theoremstyle{remark}
\newtheorem*{remark}{Remark}
\newtheorem*{solution}{Solution}
\newtheorem*{claim}{Claim}


% paragraph indent
\setlength{\parindent}{0em}
\setlength\parskip{0.5em}

\newcommand\Code{PHY5410 FA22}
\newcommand\Ass{HW06}
\newcommand\name{Haoran Sun}
\newcommand\mail{haoransun@link.cuhk.edu.cn}

\title{{\lms \Code \ \Ass}}
\author{\lms \name \ (\href{mailto:\mail}{\mail})}
\date{\sffamily \today}

\makeatletter
% \let\Title\@title
\let\theauthor\@author
\let\thedate\@date

\fancypagestyle{plain}{%
    \fancyhf{}
    \lhead{\sffamily \Ass}
    \rhead{\sffamily \name}
    \rfoot{\sffamily\thepage}

    % # 页脚自定义
    \fancyfoot[L]{
        \begin{minipage}[c]{0.06\textwidth}
            \includegraphics[height=7.5mm]{logo2.png}
        \end{minipage}
    }
}
\fancypagestyle{title}{%
    \fancyhf{}
    \renewcommand{\headrulewidth}{0pt}
    % \lhead{\Title}
    % \rhead{\theauthor}
    \rfoot{\sffamily\thepage}

    % # 页脚自定义
    \fancyfoot[L]{
        \begin{minipage}[c]{0.06\textwidth}
            \includegraphics[height=7.5mm]{logo2.png}
        \end{minipage}
    }
}
\fancyfootoffset[L]{0.3cm}

% re-define title format
\makeatletter
\renewcommand{\maketitle}{\bgroup\setlength{\parindent}{0pt}
\begin{flushleft}
  \textbf{\Large\@title}

  \@author
\end{flushleft}\egroup
}
\makeatother

\pagestyle{plain}

% lstlisting settings
\lstset{
    basicstyle=\linespread{0.7}\footnotesize,
    breaklines=true,
    basewidth=0.5em
}


\begin{document}
\maketitle
\thispagestyle{title}
% \begin{multicols*}{2}

% \begin{remark}
%     $V_\epsilon(x)$ is used to denote a $\epsilon$-neighborhood
%     \begin{align*}
%         V_\epsilon(x) = B_\epsilon(x)\setminus\{x\}
%     \end{align*}
% \end{remark}


\begin{problem}[11.3]
Define
\begin{align*}
    a_x &= \frac{1}{\sqrt[]{2}}(x - ip_x)\\
    a_y &= \frac{1}{\sqrt[]{2}}(y - ip_y)
\end{align*}
Then 
\begin{align*}
    H &= a_x^\dagger a_x + a_y^\dagger a_y + 1
\end{align*}
Compare two the 1-D harmonic oscillators, using the technique of separation of variables,
we have the wave functions for the three lowest lying energy levels:
\begin{align*}
    \phi_{0,0}(x, y) &= \frac{1}{\sqrt{\pi}}e^{-x^2/2} e^{-y^2/2}\\
    \phi_{1,0}(x, y) &= \sqrt{\frac{2}{\pi}}xe^{-x^2/2} e^{-y^2/2}\\
    \phi_{0,1}(x, y) &= \sqrt{\frac{2}{\pi}}ye^{-x^2/2} e^{-y^2/2}
\end{align*}
where $\phi_{0,1}$ and $\phi_{1,0}$ are two degenerate states.

The perturbed hamiltonian is
\begin{align*}
    H &= H_0 + \frac{1}{2}\delta H_1\\
    H_0 &= \frac{1}{2}(p_x^2+p_y^2) + \frac{1}{2}(x^2+y^2)\\
    H_1 &= xy(x^2+y^2)
\end{align*}
To avoid singularity of degenerated states, consider diagonalize the matrix
$\braket{\psi_{i,j}|H_1|\psi_{k,l}}$, $(i, j), (k, l) = (1, 0), (0, 1)$.
Consider the following integrals
\begin{align*}
    \int_{\RR^2}\d x\d y\ x^2y^2 (x^2+y^2)e^{-x^2-y^2} &= 
    \int_{\RR^2} r\d r\d \theta \sin^2\theta\cos^2\theta r^6 e^{-r^2}
    = \frac{3}{4}\pi
\end{align*}
We can verify that
\begin{align*}
    \braket{\psi_{0,0}|H_1|\psi_{0,0}} &= 0\\
    \braket{\psi_{1,0}|H_1|\psi_{1,0}} &= 0\\
    \braket{\psi_{0,1}|H_1|\psi_{0,1}} &= 0\\
    \braket{\psi_{1,0}|H_1|\psi_{0,1}} &= 
    \braket{\psi_{1,0}|H_1|\psi_{0,1}} = \frac{3}{2}
\end{align*}
Note that
\begin{align*}
    \begin{bmatrix}
        0 & 3/2 \\ 3/2 & 0
    \end{bmatrix} &= 
    \begin{bmatrix}
        1/\sqrt{2} & 1/\sqrt{2}\\
        1/\sqrt{2} & -1/\sqrt{2}
    \end{bmatrix}
    \begin{bmatrix}
        3/2 & \\
        & -3/2
    \end{bmatrix}
    \begin{bmatrix}
        1/\sqrt{2} & 1/\sqrt{2}\\
        1/\sqrt{2} & -1/\sqrt{2}
    \end{bmatrix}
\end{align*}
Define two new states to avoid singularity
\begin{align*}
    \psi_1' &= \frac{1}{\sqrt{2}}(\psi_{1,0} + \psi_{0,1})\\
    \psi_2' &= \frac{1}{\sqrt{2}}(\psi_{1,0} - \psi_{0,1})
\end{align*}
which diagonalize $H_1$.
Hence
\begin{align*}
    E_{0,0}^1 &= 0\\
    E_{1,1'}^1 &= \frac{3}{2}\\
    E_{1,2'}^1 &= -\frac{3}{2}
\end{align*}
Thus shifts are $0$, $3\delta/4$ and $-3\delta/4$, correspondingly.
Then we have first-order approximation equals to
\begin{align*}
    E_{0,0} &= 1\\
    E_{1,1'} &= 2+\frac{3}{4}\delta\\
    E_{1,2'} &= 2-\frac{3}{4}\delta
\end{align*}



\end{problem}


\begin{problem}[11.5]
Let
\begin{align*}
    H &= \frac{1}{2m}p^2 + \frac{1}{2}m\omega^2 x
\end{align*}
Then
\begin{align*}
    \psi^* H\psi &= N^2\frac{\hbar^2}{m}\mu e^{-2\mu x^2} + 
    N^2\left(\frac{1}{2}m\omega^2 - 2\frac{\hbar^2}{m}\mu^2\right)
    x^2e^{-2\mu x^2}
\end{align*}
Using the fact that
\begin{align*}
    \int_{-\infty}^{\infty} e^{-\mu x^2}\d x &= \sqrt[]{\pi}(2\mu)^{-1/2}\\
    \int_{-\infty}^{\infty} x^2e^{-\mu x^2}\d x &= \frac{1}{2}\sqrt[]{\pi}(2\mu)^{-3/2}
\end{align*}
Then the energy equals to
\begin{align*}
    E &= \frac{\braket{\psi|H|\psi}}{\braket{\psi|\psi}} = 
    \frac{\hbar^2}{2m}\mu + \frac{m\omega^2}{8}\frac{1}{\mu}
\end{align*}
Easy to derive that $E$ takes its minimum at $\mu=m\omega/2\hbar$.
Then an approximation of the ground state energy is
\begin{align*}
    E_0 &= \frac{\hbar\omega}{2}
\end{align*}


\end{problem}



\begin{problem}[11.7]
\begin{remark}
Since $\psi_0$, $\psi_1$ symmetric or antisymmetric on $\RR$, 
the energy calculated on $(0,+\infty)$ is 
the same as $(-\infty, +\infty)$.
Part (b) and (d) in this problem is solved by Mathematica.
\end{remark}
\begin{enumerate}[label=(\alph*)]
\item Note that
\begin{align*}
    H\psi_0 &= \frac{\hbar^2}{m}\kappa_0 e^{-\kappa_0x} - 
    \frac{\hbar^2}{2m}\kappa_0^2 xe^{-\kappa_0 x}
    + V(x)xe^{-\kappa_0 x}\\
    \psi_0^*H\psi_0 &= \frac{\hbar^2}{m}\kappa_0 xe^{-2\kappa_0x} - 
    \frac{\hbar^2}{2m}\kappa_0^2 x^2e^{-2\kappa_0 x}
    + V(x)x^2e^{-2\kappa_0 x}\\
\end{align*}
Since
\begin{align*}
    \int_0^\infty\d x\ e^{-ax} &= \frac{1}{a}\\
    \int_0^\infty\d x\ xe^{-ax} &= \frac{1}{a^2}\\
    \int_0^\infty\d x\ x^2e^{-ax} &= \frac{2}{a^3}
\end{align*}
we have
\begin{align*}
    \braket{\psi_0|H|\psi_0} &= 
    \frac{\hbar^2}{8\kappa_0 m} + V_0\int_0^a\d x x^2 e^{-2\kappa_0 x}
    = \frac{\hbar^2}{8\kappa_0 m} + V_0\left[
        \frac{1}{4\kappa_0^3} - e^{-2\kappa_0 a}\left(
            \frac{1}{4\kappa_0^3} + 
            \frac{2a}{4\kappa_0^2} + 
            \frac{a^2}{2\kappa_0}
        \right)
    \right]
    \\
    \braket{\psi_0|\psi_0} &= 
    \frac{1}{4\kappa_0^3}\\
    \Rightarrow
    E &= V_0 + \frac{\hbar^2}{2m}\kappa_0^2 - V_0 e^{-2\kappa_0 a}
    (1+2a\kappa_0+2a^2\kappa_0^2)
\end{align*}
Minimize $E$ w.r.t. $\kappa_0$, consider the derivative
\begin{align*}
    \frac{\partial E}{\partial\kappa_0} &= 
    \frac{\hbar^2}{m}\kappa_0 + 4V_0a^3\kappa_0^2 e^{-2\kappa_0 a} 
    = 0
\end{align*}
Then $\kappa_0$ is the positive solution of
\begin{align*}
    \kappa_0 e^{-2\kappa_0 a} &= -\frac{\hbar^2}{4V_0a^3 m}
\end{align*}

\item Using the orthogonality that $\braket{\psi_0|\psi_1}=0$, we have
\begin{align*}
    \int_{-\infty}^\infty\d x\ 
    x^2(x-n)e^{-(\kappa_0+\kappa_1)|x|} 
    &= \frac{4n}{(\kappa_0+\kappa_1)^2} = 0
    \Rightarrow n = 0
\end{align*}
Then we have
\begin{align*}
    E &= 
    V_0 + \frac{\hbar^2}{6m}\kappa_1^2
    - V_0\left[
    2a^2\kappa_1^2
    + e^{-2\kappa_1 a}\left(1 + 2a\kappa_1 + 2a^2\kappa_1^2
    + \frac{4}{3}a^3\kappa_1^3 + \frac{2}{3}a^4\kappa_1\right)
    \right]\\
    \frac{\partial}{\partial \kappa_1}E &= 
    \frac{\hbar^2}{3m}\kappa_1
    + \frac{4}{3}a^5V_0\kappa_1^4e^{-2a\kappa_1}
\end{align*}
Minimize $E_0$ w.r.t. $\kappa_1>0$, then $\kappa_1$ is the positive solution 
of
\begin{align*}
    \frac{\hbar^2}{3m}
    + \frac{4}{3}a^5V_0\kappa_1^3e^{-2a\kappa_1}
    &= 0
\end{align*}


\item I give it up\dots


\item Take $\psi_0=xe^{-\kappa_0 x^2}$, we have
\begin{align*}
    E &= -\frac{2}{\sqrt{\pi}}aV_0(2\kappa_0)^{1/2}
    + \frac{3\hbar^2}{2m}\kappa_0
    + V_0\erf [a(2\kappa_0)^{1/2}]\\
    \frac{\partial}{\partial \kappa_0} E &= 
    \frac{3\hbar^2}{2m} 
    + \frac{4}{\sqrt{\pi}}a^3V_0(2\kappa_0)^{1/2}e^{-2a^2\kappa_0}
\end{align*}
Minimize $E_0$ w.r.t. $\kappa_0>0$, then $\kappa_1$ is the positive solution 
of
\begin{align*}
    \frac{3\hbar^2}{2m} 
    + \frac{4}{\sqrt{\pi}}a^3V_0(2\kappa_0)^{1/2}e^{-2a^2\kappa_0}
    &= 0
\end{align*}


\end{enumerate}
\end{problem}





% \end{multicols*}
\end{document}

