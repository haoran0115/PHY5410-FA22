\documentclass[twoside,11pt]{article}
\usepackage[left=1in, right=1in, top=1in, bottom=1in]{geometry}
\usepackage{amsmath}
\usepackage{amssymb}
\usepackage{amsfonts}
\usepackage{mathtools}
\usepackage{amsthm}
\usepackage{fancyhdr}
\usepackage{enumitem}
\usepackage{siunitx}
\usepackage{booktabs}
\usepackage[hidelinks]{hyperref}
\usepackage{sectsty}
\usepackage{mathrsfs} % mathscr
\usepackage{tikz}
\usepackage{pgfplots}
\usepackage{multicol}
\usepackage{listings}
% \usepackage{amsart}
\usepackage{fontspec}
\usepackage{soul}


% allow H option of figure
\usepackage{float}

% math font (libertine)
\usepackage{libertinus-otf}

% braket
\usepackage{braket}

% physics
% \usepackage{physics}

% define latin modern font environment
\newcommand{\lms}{\fontfamily{lmss}\selectfont} % Latin Modern Roman
% \newcommand{\lmss}{\fontfamily{lmss}\selectfont} % Latin Modern Sans
% \newcommand{\lmss}{\fontfamily{lmtt}\selectfont} % Latin Modern Mono

% % change mathcal shape
% \usepackage[mathcal]{eucal}


% define math operators
\newcommand{\FF}{\mathbb{F}}
\newcommand{\RR}{\mathbb{R}}
\newcommand{\NN}{\mathbb{N}}
\newcommand{\ZZ}{\mathbb{Z}}
\newcommand{\QQ}{\mathbb{Q}}
\newcommand{\XX}{\mathbb{Y}}
\newcommand{\CL}{\mathcal{L}}
% \renewcommand{\d}{\mathrm{d}}
\renewcommand*\d{\mathop{}\!\mathrm{d}}
\DeclareMathOperator*{\argmax}{arg\,max}
\DeclareMathOperator*{\argmin}{arg\,min}
\DeclareMathOperator{\im}{im}
\DeclareMathOperator{\id}{id}
\DeclareMathOperator{\erf}{erf}
\renewcommand{\mod}[1]{\ (\mathrm{mod}\ #1)}

% section font style
\sectionfont{\sffamily\Large}
\subsectionfont{\sffamily\normalsize}
\subsubsectionfont{\bf}

% line spreading and break
\hyphenpenalty=5000
\tolerance=20
\setlength{\parindent}{0em}
\setlength\parskip{0.5em}
\allowdisplaybreaks
\linespread{0.9}

% enumerate settings
% no break before enumerate
\setlist[enumerate]{itemsep=2pt,topsep=2pt}

% theorem
% latex theorem
% definition style
\theoremstyle{definition}
\newtheorem{theorem}{Theorem}[subsection]
\newtheorem{axiom}{Axiom}[section]
\newtheorem{definition}{Definition}[section]
\newtheorem{example}{Example}[section]
\newtheorem{question}{Question}[section]
\newtheorem{exercise}{Exercise}[section]
\newtheorem*{exercise*}{Exercise}
\newtheorem{lemma}{Lemma}[section]
\newtheorem{proposition}{Proposition}[section]
\newtheorem{corollary}{Corollary}[section]
\newtheorem*{theorem*}{Theorem}
\newtheorem{problem}{Problem}
% remark style
\theoremstyle{remark}
\newtheorem*{remark}{Remark}
\newtheorem*{solution}{Solution}
\newtheorem*{claim}{Claim}


% paragraph indent
\setlength{\parindent}{0em}
\setlength\parskip{0.5em}

\newcommand\Code{PHY5410 FA22}
\newcommand\Ass{HW11}
\newcommand\name{Haoran Sun}
\newcommand\mail{haoransun@link.cuhk.edu.cn}

\title{{\lms \Code \ \Ass}}
\author{\lms \name \ (\href{mailto:\mail}{\mail})}
\date{\sffamily \today}

\makeatletter
% \let\Title\@title
\let\theauthor\@author
\let\thedate\@date

\fancypagestyle{plain}{%
    \fancyhf{}
    \lhead{\sffamily \Ass}
    \rhead{\sffamily \name}
    \rfoot{\sffamily\thepage}

    % # 页脚自定义
    \fancyfoot[L]{
        \begin{minipage}[c]{0.06\textwidth}
            \includegraphics[height=7.5mm]{logo2.png}
        \end{minipage}
    }
}
\fancypagestyle{title}{%
    \fancyhf{}
    \renewcommand{\headrulewidth}{0pt}
    % \lhead{\Title}
    % \rhead{\theauthor}
    \rfoot{\sffamily\thepage}

    % # 页脚自定义
    \fancyfoot[L]{
        \begin{minipage}[c]{0.06\textwidth}
            \includegraphics[height=7.5mm]{logo2.png}
        \end{minipage}
    }
}
\fancyfootoffset[L]{0.3cm}

% re-define title format
\makeatletter
\renewcommand{\maketitle}{\bgroup\setlength{\parindent}{0pt}
\begin{flushleft}
  \textbf{\Large\@title}

  \@author
\end{flushleft}\egroup
}
\makeatother

\pagestyle{plain}

% lstlisting settings
\lstset{
    basicstyle=\linespread{0.7}\footnotesize,
    breaklines=true,
    basewidth=0.5em
}


\begin{document}
\maketitle
\thispagestyle{title}
% \begin{multicols*}{2}

% \begin{remark}
%     $V_\epsilon(x)$ is used to denote a $\epsilon$-neighborhood
%     \begin{align*}
%         V_\epsilon(x) = B_\epsilon(x)\setminus\{x\}
%     \end{align*}
% \end{remark}

\begin{problem}[1.1]
Denote
\begin{align*}
    \ket{i_1,\dots,i_N} &= 
    \varphi_{i_1}(x_1)\cdots\varphi_{i_N}(x_N)
\end{align*}
Assume this basis is complete
\begin{align*}
    \sum_{i_1,\dots,i_N}
    \ket{i_1,\dots,i_N}
    \bra{i_1,\dots,i_N} = 1
\end{align*}
Then $\forall \psi_{s/a}$, we have
\begin{align*} 
    \sum_{i_1,\dots,i_N}
    \ket{i_1,\dots,i_N}
    \braket{i_1,\dots,i_N|\psi_{s/a}} &= \psi_{s/a}\\
    \sum_{i_1,\dots,i_N}
    \frac{1}{\sqrt{N!}}S_\pm\ket{i_1,\dots,i_N}
    \braket{i_1,\dots,i_N|\psi_{s/a}} 
    &= \frac{1}{\sqrt{N!}}S_\pm\psi_{s/a}\\
    \sum_{i_1,\dots,i_N}
    \frac{1}{\sqrt{N!}}S_\pm\ket{i_1,\dots,i_N}
    \braket{i_1,\dots,i_N|\frac{1}{N!}
    S_\pm^\dagger S_\pm\psi_{s/a}}
    &= \psi_{s/a}\\
    \sum_{i_1,\dots,i_N}
    \frac{1}{N!}S_\pm\ket{i_1,\dots,i_N}
    \braket{i_1,\dots,i_N|S_\pm^\dagger\psi_{s/a}}
    &= \psi_{s/a}
\end{align*}
which shows that $S_\pm\ket{i_1,\dots,i_N}$ complete
$\forall \psi_{s/a}$.

\end{problem}


\begin{problem}[1.3]\
\begin{enumerate}[label=(\alph*)]
\item Since $[a, (a^\dagger)^m] = m(a^\dagger)^{m-1}$
\begin{align*}
    ae^{\alpha a^\dagger} &= 
    a\sum_{n}\frac{1}{n!}(\alpha a^\dagger)^n =
    \frac{\alpha^n}{n!}a(a^\dagger)^n = 
    \sum_n \frac{\alpha^n}{n!}
    \left[
        n (a^\dagger)^{n-1}  + (a^\dagger)^n a
    \right]
    = \alpha e^{\alpha a^\dagger} + e^{\alpha a^\dagger} a
    \Rightarrow [a, e^{\alpha a^\dagger}] = 
    \alpha e^{\alpha a^\dagger}
\end{align*}

\item Note that
\begin{align*}
    e^{-\alpha a^\dagger} a e^{\alpha a^\dagger} &= 
    e^{-\alpha a^\dagger}[e^{\alpha a^\dagger}a + \alpha e^{\alpha a^\dagger}]
    = a + \alpha
\end{align*}

\item Note that
\begin{align*}
    e^{-\alpha a^\dagger}\beta a e^{\alpha a^\dagger}
    &= 
    \beta a + \beta\alpha
    \Rightarrow
    e^{-\alpha a^\dagger}(\beta a)^n e^{\alpha a^\dagger}
    = (\beta a + \beta\alpha)^n
\end{align*}
Hence
\begin{align*}
    e^{-\alpha a^\dagger} e^{\beta \alpha}
    e^{\alpha a^\dagger} &= 
    \sum_n\frac{1}{n!}(\beta a + \beta \alpha)^n 
    = e^{\beta a + \beta\alpha}
    = e^{\beta a}e^{\beta\alpha}
\end{align*}

\item Since
\begin{align*}
    e^{\alpha a^\dagger a}a &=
    \sum_n \frac{1}{n!}\alpha^n (a^\dagger a)^n a =
    \sum_n \frac{1}{n!}\alpha^n (a^\dagger a)^{n-1} a (a^\dagger a - 1) =
    \sum_n \frac{1}{n!}\alpha^n a(a^\dagger a - 1)^n 
    = ae^{\alpha(a^\dagger a- 1)}
\end{align*}
Hence
\begin{align*}
    e^{\alpha a^\dagger a}a e^{-\alpha a^\dagger a}
    = ae^{\alpha a^\dagger a}e^{-\alpha} a e^{-\alpha a^\dagger a}
    &= ae^{-\alpha}
\end{align*}

\end{enumerate}
\label{prob:q2}
\end{problem}


\begin{problem}[1.4]
There are two methods to solve the problem
\begin{enumerate}[label=(\roman*)]
\item Using the differential relation
\begin{align*}
    \frac{\d}{\d t} a_i(t) &= 
    \frac{iHt}{\hbar}e^{iHt/\hbar}a_ie^{-iHt/\hbar}
    - e^{iHt/\hbar}a_i \frac{iHt}{\hbar}e^{-iHt/\hbar}\\
    &= 
    \frac{iHt}{\hbar}e^{iHt/\hbar}a_ie^{-iHt/\hbar}
    -\frac{iHt}{\hbar}e^{iHt/\hbar}a_ie^{-iHt/\hbar}
    - \frac{iHt}{\hbar}e^{iHt/\hbar}\frac{i\epsilon_i t}{\hbar}a_ie^{-iHt/\hbar}\\
    &= -\frac{i\epsilon_i t}{\hbar}a_i\\
    \Rightarrow
    a_i(t) &= a_i(0) e^{-i\epsilon_it/\hbar} = a_i e^{-i\epsilon_it/\hbar}
\end{align*}

\item Using the Bose commutation relation $[a_i, a_j^\dagger] = \delta_{ij}$ and the equation
from Problem \ref{prob:q2},
we have
\begin{align*}
    a_i(t) &= e^{i\epsilon_it a_i^\dagger a_i/\hbar} 
    a_i e^{-i\epsilon_it a_i^\dagger a_i/\hbar}
    = a_i e^{-i\epsilon_it/\hbar}
\end{align*}

\end{enumerate}

\end{problem}




% \end{multicols*}
\end{document}

