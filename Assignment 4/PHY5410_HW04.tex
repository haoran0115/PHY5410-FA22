\documentclass[twoside,11pt]{article}
\usepackage[left=1in, right=1in, top=1in, bottom=1in]{geometry}
\usepackage{amsmath}
\usepackage{amssymb}
\usepackage{amsfonts}
\usepackage{mathtools}
\usepackage{amsthm}
\usepackage{fancyhdr}
\usepackage{enumitem}
\usepackage{siunitx}
\usepackage{booktabs}
\usepackage[hidelinks]{hyperref}
\usepackage{sectsty}
\usepackage{mathrsfs} % mathscr
\usepackage{tikz}
\usepackage{pgfplots}
\usepackage{multicol}
\usepackage{listings}
% \usepackage{amsart}
\usepackage{fontspec}


% allow H option of figure
\usepackage{float}

% math font (libertine)
\usepackage{libertinus-otf}

% braket
\usepackage{braket}

% define latin modern font environment
\newcommand{\lms}{\fontfamily{lmss}\selectfont} % Latin Modern Roman
% \newcommand{\lmss}{\fontfamily{lmss}\selectfont} % Latin Modern Sans
% \newcommand{\lmss}{\fontfamily{lmtt}\selectfont} % Latin Modern Mono

% % change mathcal shape
% \usepackage[mathcal]{eucal}


% define math operators
\newcommand{\FF}{\mathbb{F}}
\newcommand{\RR}{\mathbb{R}}
\newcommand{\NN}{\mathbb{N}}
\newcommand{\ZZ}{\mathbb{Z}}
\newcommand{\QQ}{\mathbb{Q}}
\newcommand{\XX}{\mathbb{Y}}
\newcommand{\CL}{\mathcal{L}}
% \renewcommand{\d}{\mathrm{d}}
\renewcommand*\d{\mathop{}\!\mathrm{d}}
\DeclareMathOperator*{\argmax}{arg\,max}
\DeclareMathOperator*{\argmin}{arg\,min}
\DeclareMathOperator{\im}{im}
\DeclareMathOperator{\id}{id}
\renewcommand{\mod}[1]{\ (\mathrm{mod}\ #1)}

% section font style
\sectionfont{\sffamily\Large}
\subsectionfont{\sffamily\normalsize}
\subsubsectionfont{\bf}

% line spreading and break
\hyphenpenalty=5000
\tolerance=20
\setlength{\parindent}{0em}
\setlength\parskip{0.5em}
\allowdisplaybreaks
\linespread{0.9}

% theorem
% latex theorem
% definition style
\theoremstyle{definition}
\newtheorem{theorem}{Theorem}[subsection]
\newtheorem{axiom}{Axiom}[section]
\newtheorem{definition}{Definition}[section]
\newtheorem{example}{Example}[section]
\newtheorem{question}{Question}[section]
\newtheorem{exercise}{Exercise}[section]
\newtheorem*{exercise*}{Exercise}
\newtheorem{lemma}{Lemma}[section]
\newtheorem{proposition}{Proposition}[section]
\newtheorem{corollary}{Corollary}[section]
\newtheorem*{theorem*}{Theorem}
\newtheorem{problem}{Problem}
% remark style
\theoremstyle{remark}
\newtheorem*{remark}{Remark}
\newtheorem*{solution}{Solution}
\newtheorem*{claim}{Claim}


% paragraph indent
\setlength{\parindent}{0em}
\setlength\parskip{0.5em}

\newcommand\Code{PHY5410 FA22}
\newcommand\Ass{HW04}
\newcommand\name{Haoran Sun}
\newcommand\mail{haoransun@link.cuhk.edu.cn}

\title{{\lms \Code \ \Ass}}
\author{\lms \name \ (\href{mailto:\mail}{\mail})}
\date{\sffamily \today}

\makeatletter
% \let\Title\@title
\let\theauthor\@author
\let\thedate\@date

\fancypagestyle{plain}{%
    \fancyhf{}
    \lhead{\sffamily \Ass}
    \rhead{\sffamily \name}
    \rfoot{\sffamily\thepage}

    % # 页脚自定义
    \fancyfoot[L]{
        \begin{minipage}[c]{0.06\textwidth}
            \includegraphics[height=7.5mm]{logo2.png}
        \end{minipage}
    }
}
\fancypagestyle{title}{%
    \fancyhf{}
    \renewcommand{\headrulewidth}{0pt}
    % \lhead{\Title}
    % \rhead{\theauthor}
    \rfoot{\sffamily\thepage}

    % # 页脚自定义
    \fancyfoot[L]{
        \begin{minipage}[c]{0.06\textwidth}
            \includegraphics[height=7.5mm]{logo2.png}
        \end{minipage}
    }
}
\fancyfootoffset[L]{0.3cm}

% re-define title format
\makeatletter
\renewcommand{\maketitle}{\bgroup\setlength{\parindent}{0pt}
\begin{flushleft}
  \textbf{\Large\@title}

  \@author
\end{flushleft}\egroup
}
\makeatother

\pagestyle{plain}

% lstlisting settings
\lstset{
    basicstyle=\linespread{0.7}\footnotesize,
    breaklines=true,
    basewidth=0.5em
}


\begin{document}
\maketitle
\thispagestyle{title}
% \begin{multicols*}{2}

% \begin{remark}
%     $V_\epsilon(x)$ is used to denote a $\epsilon$-neighborhood
%     \begin{align*}
%         V_\epsilon(x) = B_\epsilon(x)\setminus\{x\}
%     \end{align*}
% \end{remark}

\begin{problem}[6.3]
Using the relation that $u(r) = rR(r)$, $\varrho=\kappa r$
(more strictly, we should use the notation $v(\varrho)=u(r)=u(\varrho/\kappa)$ rather than $u(\varrho)$
to denote $u(\varrho/\kappa)$),
define the following notations
\begin{align*}
    \braket{r^k} &= 4\pi \int r^2\d r\ r^k R(r)^2\\
    \braket{\varrho^k}_u &= 4\pi\int\d \varrho\ \varrho^k u(\varrho)^2
\end{align*}
one can verify
\begin{align*}
    \braket{\varrho^k}_u &= 4\pi\int\d \kappa r\ (\kappa r)^k r^2 R(r)^2
    =4\pi\kappa^{k+1} \int r^2\d ri\ r^kR^r(r) = \kappa^{k+1}\braket{r^k}
\end{align*}
Using the differential equation
\begin{align}
    \left[
        \frac{\d^2}{\d\varrho^2} - \frac{l(l+1)}{\varrho^2} + \frac{2n}{\varrho} - 1
    \right]u(\varrho) = 0
\end{align}
Multiply $\varrho^{k+1}u'_{nl}(\varrho)$ on the left, we get
\begin{align*}
    \varrho^{k+1}u'u'' - \varrho^{k-1}l(l+1)uu' + 2n\varrho^k uu' - \varrho^{k+1}uu' = 0
\end{align*}
Using the fact that
\begin{align*}
    4\pi\int\d\varrho\ \varrho^m uu' &= -\frac{m}{2}4\pi\int\d\varrho\ \varrho^{m-1}u^2
    = -\frac{m}{2}\braket{\varrho^{m-1}}_u\\
    4\pi\int\d\varrho\ \varrho^m u'u'' &= -2\pi(m-1)\int\d\varrho\ \varrho^{m-1} (u')^2\\
    4\pi\int\d\varrho\ \varrho^m uu'' &= -4\pi m\int\d\varrho\ \varrho^{m-1}uu' -4\pi\int\d\varrho\
    \varrho^m (u')^2
\end{align*}
we can multiply the expression by $4\pi$ and integrate it
\begin{align}
    -2\pi (k+1)\int\d\varrho\ \varrho^k (u')^2 + \frac{1}{2}(k-1)l(l+1)\braket{\varrho^{k-2}}_u
    - nk\braket{\varrho^{k-1}}_u + \frac{1}{2}(k+1)\braket{\varrho^k}_u = 0
    \label{eq:1}
\end{align}
we can also multiply the equation by $\varrho^k u_{nl}(\varrho)$
\begin{align*}
    \varrho^kuu'' - l(l+1)\varrho^{k-2}u^2 + 2n\varrho^{k-1}u^2 - \varrho^k u^2 = 0
\end{align*}
multiply this equation by $4\pi$ and integrate it
\begin{align}
    \frac{1}{2}k(k-1)\braket{\varrho^{k-2}}_u -4\pi\int\d\varrho\ \varrho^k (u')^2
    - l(l+1)\braket{\varrho^{k-2}}_u + 2n\braket{\varrho^{k-1}}_u
    - \braket{\varrho^k}_u = 0
    \label{eq:2}
\end{align}
Combing equation \ref{eq:1} and \ref{eq:2} to eliminate $(u')^2$ term, we can get
\begin{align*}
    [\frac{1}{2}(k-1)l(l+1) - \frac{1}{4}k(k^2-1)+\frac{1}{2}l(l+1)(k+1)]\braket{\varrho^{k-2}}_u 
    - [nk+n(k+1)]\braket{\varrho^{k-1}}_u + (k+1)\braket{\varrho^k}_u &= 0\\
    \frac{k}{4}[4l(l+1)-(k^2-1)]\braket{\varrho^{k-2}}_u - n(2k+1)\braket{\varrho^{k-1}}_u
    +(k+1)\braket{\varrho^k}_u &= 0\\
    \frac{k}{4}[(2l+1)^2 - k^2]\braket{\varrho^{k-2}}_u - n(2k+1)\braket{\varrho^{k-1}}_u
    + (k+1)\braket{\varrho^k}_u &= 0
\end{align*}
Plugin $\kappa = 1/na$ and $\braket{\varrho^k}_u = \kappa^{k+1}\braket{r^k}$, we get
\begin{align*}
    \frac{k}{4}[(2l+1)^2 - k^2]\braket{r^{k-2}} - n(2k+1)\kappa\braket{r^{k-1}}
    + (k+1)\kappa^2\braket{r^k} &= 0\\
    \Rightarrow \frac{k+1}{n^2}\braket{r^k} - (2k+1)a\braket{r^{k-1}} + \frac{k}{4}[(2l+1)^2 - k^2]a^2\braket{r^{k-2}}
    &= 0
\end{align*}


\end{problem}

\begin{problem}[6.13]
Let $A_i$ denotes the $i$th component of $\mathbf{A}$, then 
\begin{align*}
    A_i &= \frac{1}{2m}(
        \epsilon_{ijk}p_jL_k - \epsilon_{ijk}L_jp_k
    ) - \frac{Ze^2 }{r}x_i = \frac{1}{2m}\epsilon_{ijk}
    (p_jL_k + L_kp_j) - \frac{Ze^2}{r}x_i
\end{align*}
where the summation would be taken in place of $j$ and $k$.
Since $p_j^\dagger=p_j$, $L_k^\dagger=L_k$, then we can prove that
$A_i$ is hermitian
\begin{align*}
    A_i^\dagger  = 
    \frac{1}{2m}\epsilon_{ijk}
    (L_k^\dagger p_j^\dagger + p_j^\dagger L_k^\dagger) - \frac{Ze^2}{r}x_i^\dagger
    =
    \frac{1}{2m}\epsilon_{ijk}
    (L_kp_j+ p_jL_k) - \frac{Ze^2}{r}x_i = A_i
\end{align*}
Therefore $\mathbf{A}$ is hermitian.

To prove $[\mathbf{A}, H]=0$, we can show that $[A_i,H]=0$.
$A_i$ could be simplified to the following expression by direct summation
over $j$ and $k$.
\begin{align*}
    A_i &= \frac{1}{2m}\epsilon_{ijk}\epsilon_{kmn}
    (p_jx_mp_n+x_mp_np_j) - \frac{Ze^2}{r}x_i\\
    &= \frac{1}{2m}\epsilon_{ijk}\epsilon_{kmn}
    [(x_mp_jp_n - i\hbar\delta_{jm}p_n + x_mp_np_j)] - \frac{Ze^2}{r}x_i\\
    &= \frac{1}{2m}\epsilon_{kij}\epsilon_{kmn}
    (2x_mp_jp_n - i\hbar\delta_{jm}p_n) - \frac{Ze^2}{r}x_i\\
    &= \frac{1}{2m}(\delta_{im}\delta_{jn} - \delta_{in}\delta_{jm})
    (2x_mp_jp_n - i\hbar\delta_{jm}p_n)
\end{align*}
Since
\begin{align*}
    \delta_{im}\delta_{jn}x_mp_jp_n &= \delta_{jn}x_ip_jp_n = x_ip^2\\
    \delta_{in}\delta_{jm}x_mp_jp_n &= \delta_{jm}x_mp_ip_j = (x\cdot p)p_i\\
    \delta_{im}\delta_{jm}\delta_{jn}i\hbar p_n&= 
    \delta_{ij}\delta_{jn}i\hbar p_n = i\hbar p_i\\
    \delta_{in}\delta_{jm}^2p_n &= \delta_{jm}^2 i\hbar p_i = 3i\hbar p_i
\end{align*}
Then we have 
\begin{align*}
    A_i &= \frac{1}{m}x_ip^2 - \frac{1}{m}(x\cdot p)p_i
    + \frac{1}{m}i\hbar p_i - \frac{Ze^2}{r}x_i
\end{align*}
Using the fact that $H=p^2/2m - Ze^2/r$, we have
\begin{align*}
    A_iH &=
    \frac{1}{2m^2}[x_i(p^2)^2 - (x\cdot p) p^2p_i + i\hbar p^2p_i]
    -\frac{Ze^2}{2m}\frac{1}{r}x_ip^2 - 
    \frac{Ze^2}{m}[x_ip^2\frac{1}{r} - (x\cdot p)p_i\frac{1}{r}
    +i\hbar p_i\frac{1}{r}]
    + \frac{Ze^4}{r^2}x_i\\
    HA_i &= 
    \frac{1}{2m^2}[p^2x_ip^2 - p^2(x\cdot p)p_i + i\hbar p^2p_i]
    -\frac{Ze^2}{2m}p^2\frac{1}{r}x_i 
     - \frac{Ze^2}{m}[\frac{1}{r}x_ip^2 - \frac{1}{r}(x\cdot p)p_i + 
    \frac{1}{r}i\hbar p_i]
    + \frac{Ze^4}{r^2}x_i
\end{align*}
Using the commutation relationship
\begin{align*}
    p^2 x_i &= x_ip^2 - 2i\hbar p_i\\
    p^2(x\cdot p) &= \sum_j p^2x_jp_j 
    = \sum_j x_jp^2p_j - 2i\hbar p_j^2= 
    (x\cdot p)p^2 - 2i\hbar p^2
\end{align*}
Then it is easy to show that
\begin{align*}
    p^2x_ip^2 - p^2(x\cdot p)p_i &= 
    x_i(p^2)^2 - 2i\hbar p^2p_i 
    -(x\cdot p)p^2p_i + 2i\hbar p^2 p_i=
    x_i(p^2)^2 - (x\cdot p) p^2p_i
\end{align*}
Moreover, using the fact that
\begin{align*}
    p_i\frac{1}{r} &= i\hbar\frac{1}{r^3}x_i + \frac{1}{r}p_i\\
    p_ip_j\frac{1}{r} &= 
    i\hbar\frac{1}{r^3}x_ip_j + i\hbar\frac{1}{r^3}x_jp_i + 
    \frac{1}{r}p_ip_j + \hbar^2
    \frac{\delta_{ij}r^2 - 3x_ix_j}{r^5}\\
    p^2\frac{1}{r} &= 2i\hbar\frac{1}{r^3}(x\cdot p) + \frac{1}{r}p^2\\
    p^2x_i\frac{1}{r} &= x_ip^2\frac{1}{r} - 2i\hbar p_i\frac{1}{r}
    =2i\hbar\frac{1}{r^3}x_i(x\cdot p) + \frac{1}{r}x_ip^2
    +2\hbar^2\frac{1}{r^3}x_i - 2i\hbar\frac{1}{r}p_i
\end{align*}
we have
\begin{align*}
    -\frac{Ze^2}{2m}\frac{1}{r}x_ip^2 - 
    \frac{Ze^2}{m}[x_ip^2\frac{1}{r} - (x\cdot p)p_i\frac{1}{r}
    +i\hbar p_i\frac{1}{r}] &= 
    -\frac{Ze^2}{2m}\frac{1}{r}x_ip^2 - \frac{Ze^2}{m}\Big[
        2i\hbar\frac{1}{r^3}x_i(x\cdot p) + \frac{1}{r}x_ip^2\\
        &-\ i\hbar\frac{1}{r^3}x_i(x\cdot p) - i\hbar\frac{1}{r}p_i 
        - \frac{1}{r}(x\cdot p)p_i + 2\hbar^2\frac{1}{r^3}x_i\\
        &-\ \hbar^2\frac{1}{r^3}x_i + i\hbar\frac{1}{r}p_i
    \Big]\\
    &= -\frac{Ze^2}{2m}\frac{1}{r}x_ip_i - \frac{Ze^2}{m}\Big[
        \frac{1}{r}x_ip^2 - \frac{1}{r}(x\cdot p)p_i\\
    &+\ i\hbar\frac{1}{r^3}x_i(x\cdot p) + \hbar^2\frac{1}{r^3}x_i
    \Big]\\
    -\frac{Ze^2}{2m}p^2\frac{1}{r}x_i 
     - \frac{Ze^2}{m}[\frac{1}{r}x_ip^2 - \frac{1}{r}(x\cdot p)p_i + 
    \frac{1}{r}i\hbar p_i] &= 
    -\frac{Ze^2}{2m}\frac{1}{r}x_i p^2 - \frac{Ze^2}{m}i\hbar\frac{1}{r^3}x_i(x\cdot p)
    -\frac{Ze^2}{m}\hbar^2\frac{1}{r^3}x_i\\
    &+\ \frac{Ze^2}{m}i\hbar\frac{1}{r}p_i - \frac{Ze^2}{m}\Big[\frac{1}{r}x_ip^2 - \frac{1}{r}(x\cdot p)p_i + i\hbar\frac{1}{r}p_i\Big]\\
    &= -\frac{Ze^2}{2m}\frac{1}{r}x_ip^2 - \frac{Ze^2}{m}\Big[\frac{1}{r}x_ip^2 - \frac{1}{r}(x\cdot p)p_i\\
    &+\ i\hbar\frac{1}{r^3}x_i(x\cdot p) + \hbar^2\frac{1}{r^3}x_i\Big]
\end{align*}
Therefore $A_iH = HA_i$, hence $[A_i, H] = 0$, we have $[\mathbf{A}, H] = 0$.

The proof of $\mathbf{A}\cdot\mathbf{L} = \mathbf{L}\cdot\mathbf{A}=0$
begins from the following claims.
\begin{claim}
    $\mathbf{L}\cdot x = x\cdot \mathbf{L} = 0$.
\end{claim}
\begin{proof}
    Note that
    \begin{align*}
        x\cdot\mathbf{L} &= \sum_{ijk}\epsilon_{ijk}x_ix_jp_k = \sum_k \left(\sum_{ij}\epsilon_{ijk}x_ix_j\right)p_k
        = \sum_k 0p_k = 0\\
        \mathbf{L}\cdot x &= \sum_{ijk}\epsilon_{ijk}x_jp_kx_i = 
        \sum_{ijk}\epsilon_{ijk}x_j(x_ip_k-i\hbar\delta_{ik}) =
        \sum_{ijk}\epsilon_{ijk} x_ix_jp_k-i\hbar\epsilon_{ijk}\delta_{ik}x_j = 0
        \qedhere
    \end{align*}
\end{proof}
\begin{claim}
    $\mathbf{L}\cdot (p\times\mathbf{L} - \mathbf{L}\times p) = 
    (p\times \mathbf{L} - \mathbf{L}\times p)\cdot \mathbf{L} = 0$.
\end{claim}
\begin{proof}
    Note that
    \begin{align*}
        \mathbf{L}\cdot (p\times\mathbf{L} - \mathbf{L}\times p) &= 
        \sum_{ijk}\epsilon_{ijk}L_ip_jL_k - \epsilon_{ijk}L_iL_jp_k
        =\sum_{ijk}\epsilon_{ijk}L_i(p_jL_k + L_kp_j)\\
        &= \sum_{ijk}\epsilon_{ijk}(2L_iL_kp_j + i\hbar\sum_{l}\epsilon_{kjl}L_ip_l)\\
        &= \sum_{ijk}2\epsilon_{ijk}L_iL_kp_j - i\hbar\epsilon_{ijk}\epsilon_{kjl}L_ip_l\\
        &= -\sum_j 2i\hbar L_jp_j - \sum_ki\hbar L_kp_k + \sum_i 3i\hbar L_ip_i\\
        &= 0\\
        (p\times\mathbf{L} - \mathbf{L}\times p)\cdot\mathbf{L} &=
        \sum_{ijk}\epsilon_{ijk}(p_jL_k+L_kp_j)L_i\\
        &= \sum_{ijk}\epsilon_{ijk}(2p_jL_k + i\hbar\sum_l\epsilon_{kjl}p_l)L_i\\
        &= \sum_j 2i\hbar p_j L_j + \sum_ki\hbar p_k L_k - 3\sum_i i\hbar p_iL_i\\
        &= 0\qedhere
    \end{align*}
\end{proof}
Therefore, it is easy to show that
\begin{align*}
    \mathbf{L}\cdot\mathbf{A} &= \frac{1}{2m}\mathbf{L}\cdot(p\times\mathbf{L}
    -\mathbf{L}\times p) - (\mathbf{L}\times x)\frac{Ze^2}{r} = 0\\
    \mathbf{A}\cdot\mathbf{L} &= \frac{1}{2m}(p\times\mathbf{L}-\mathbf{L}\times p)\cdot\mathbf{L}
    - \frac{Ze^2}{r}(x\times\mathbf{L}) = 0
\end{align*}

\end{problem}




\begin{problem}[7.1]
Since
\begin{align*}
    \nabla\cdot\mathbf{j} &= \frac{1}{2m}\left[
        \psi^*(-i\hbar\nabla^2\psi-\frac{e}{c}A\cdot\nabla\psi - \frac{e}{c}(\nabla\cdot\mathbf{A})\psi)
        + (\nabla\psi^*)\cdot(-i\hbar\nabla\psi - \frac{e}{c}\mathbf{A}\psi) 
    \right]+ \mathrm{c.c.} \\ 
    &= \frac{1}{2m}\left(
    -i\hbar\psi^*\nabla^2\psi - \frac{e}{c}\psi^*\mathbf{A}\cdot\nabla\psi 
    -\frac{e}{c}(\nabla\cdot\mathbf{A})\psi^*\psi
    - i\hbar\nabla\psi^*\cdot\nabla\psi
    - \frac{e}{c}\nabla\psi^*\cdot\mathbf{A}\psi 
    \right)+ \mathrm{c.c.} \\ 
    &= \frac{1}{2m}\left(
    -i\hbar\psi^*\nabla^2\psi - \frac{e}{c}\psi^*\mathbf{A}\cdot\nabla\psi 
    -\frac{e}{c}(\nabla\cdot\mathbf{A})\psi^*\psi
    - \frac{e}{c}\nabla\psi^*\cdot\mathbf{A}\psi 
    \right)+ \mathrm{c.c.} \\ 
    &= \frac{1}{2m}\left(
    -i\hbar\psi^*\nabla^2\psi - \frac{2e}{c}\psi^*\mathbf{A}\cdot\nabla\psi 
    -\frac{e}{c}(\nabla\cdot\mathbf{A})\psi^*\psi 
    \right) + \mathrm{c.c.}
\end{align*}
and
\begin{align*}
    \frac{\partial}{\partial t}\psi^*\psi &= \psi^*\frac{\partial}{\partial t}\psi + \mathrm{c.c.}\\
    &= \frac{1}{i\hbar}\psi^* H\psi + \mathrm{c.c.}\\
    &= \frac{1}{i\hbar}\psi^*\left(
        -\frac{\hbar^2}{2m}\nabla^2\psi + \frac{i\hbar e}{mc}\mathbf{A}\cdot\nabla\psi
        + \frac{i\hbar e}{2mc}(\nabla\cdot \mathbf{A})\psi + \frac{e^2}{2mc}\mathbf{A}^2\psi
        +e\Phi\psi
    \right) + \mathrm{c.c.}\\
    &= \frac{1}{2m}\left(
        i\hbar\psi^*\nabla^2\psi + \frac{2e}{c}\psi^*\mathbf{A}\cdot\nabla\psi 
        + \frac{e}{c}(\nabla\cdot\mathbf{A})\psi^*\psi
    \right) + \mathrm{c.c.}
\end{align*}
Hence $\partial_t \psi^*\psi + \nabla\cdot\mathbf{j} = 0$.
\end{problem}




\begin{problem}[7.3]
Let the equation be $H\psi = \lambda\psi$, where the hamiltonian $H$ equals to
\begin{align*}
    H &= \frac{1}{2m} \left[p_x^2 + \left(p_y - \frac{e}{c}Bx\right)^2 + p_z^2
    \right] - eEx
\end{align*}
Suppose the solution in the form of $\psi(x, y, z) = e^{ik_2y}e^{ik_3z}\psi(x)$, then we can get
a hamiltonian $H_x$ only related to $x$
\begin{align*}
    H_x &=
    \frac{1}{2m}\left[
    p_x^2 + \left(\hbar k_2 - \frac{e}{c}Bx\right)^2 + \hbar^2k_3^2
    \right] - eEx\\
    &= \frac{1}{2m}\left[
    p_x^2 + \left(-\hbar k_2 + \frac{e}{c}Bx\right)^2 - 2meEx + \hbar^2k_3^2
    \right]\\
    &= \frac{1}{2m}\left[
    p_x^2 + \left(\frac{e}{c}Bx + (-\hbar k_2 - \frac{mcE}{B})\right)^2
    - (-\hbar k_2-\frac{mcE}{B})^2
    +\hbar^2(k_2^2+k_3^2)
    \right]
\end{align*}
Borrow the idea from a one-dimensional harmonic oscillator where
\begin{align*}
    H &= \frac{1}{2m}[p^2 + (m\omega x)^2]\\
    a_\pm &= \frac{1}{\sqrt{2\hbar m\omega}}(m\omega x \mp ip)\\
    \psi_0(x) &= \left(\frac{m\omega}{\pi\hbar}\right)^{1/4}e^{-\frac{m\omega}{2\hbar}x^2}\\
    \psi_n(x) &= A_n(a_+)^n\psi_0(x) \text{ with } E_n = \left(n + \frac{1}{2}\right)\hbar\omega
\end{align*}
then we can define $\omega_B$, $x'$ and some operators
\begin{align*}
    \omega_B &=  \frac{eB}{mc}\\
    x' &= x + \frac{c}{eB}(-\hbar k_2 - \frac{mcE}{B})\\
    p' &= p_{x'} = p_x\\
    a_\pm &= \frac{1}{\sqrt{2\hbar m\omega_B}}(m\omega_B x' \mp ip')
\end{align*}
Then we have the solution to $\psi(x')$ and $\psi(x, y, z)$
\begin{align*}
    \psi_0(x') &= \left(\frac{m\omega_B}{\pi\hbar}\right)^{1/4}e^{-\frac{m\omega_B}{2\hbar}x'^2}
    = \left(\frac{m\omega_B}{\pi\hbar}\right)^{1/4}e^{-\frac{m\omega_B}{2\hbar}(x+\frac{c}{eB}(-\hbar
    k_2 - mcE))^2}\\
    \psi_n(x') &= A_n(a_+)^n\psi_0(x') \text{ with } E_n = \left(n + \frac{1}{2}\right)\hbar\omega_B + C\\
    \psi(x, y, z) &= e^{ik_2y}e^{ik_3z}
    \left(\frac{m\omega_B}{\pi\hbar}\right)^{1/4}e^{-\frac{m\omega_B}{2\hbar}(x+\frac{c}{eB}(-\hbar
    k_2 - mcE))^2}\\
    &= e^{ik_2y}e^{ik_3z}
    \left(\frac{m\omega_B}{\pi\hbar}\right)^{1/4}e^{-\frac{m\omega_B}{2\hbar}(x-x_0)^2}
\end{align*}
where $A_n$ is the normalization constant and 
\begin{align*}
    C &= \frac{1}{2m}\left[
        - (-\hbar k_2-\frac{mcE}{B})^2
        +\hbar^2(k_2^2+k_3^3)
    \right]
\end{align*}

Let $L_x$ denote the restriction on $x$ 
and $L_y$ denote the periodic condition on $y$.
Then $e^{ik_2y}$ should have period $L_y$, which means
\begin{align*}
    k_2 &= \frac{2n\pi}{L_y},\ n\in\NN
\end{align*}
Also, the center $x_0$ satisfies $0 \leq x_0=\hbar ck_2/eB\leq L_x$, then we have
\begin{align*}
    \frac{\hbar c}{eB}\frac{2n\pi}{L_y} \leq L_x
    \Rightarrow
    n \leq \frac{L_xL_yBe}{2\pi\hbar c} = N
\end{align*}
then $N$ is the degeneracy.





\end{problem}







% \end{multicols*}
\end{document}

