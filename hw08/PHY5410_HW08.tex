\documentclass[twoside,11pt]{article}
\usepackage[left=1in, right=1in, top=1in, bottom=1in]{geometry}
\usepackage{amsmath}
\usepackage{amssymb}
\usepackage{amsfonts}
\usepackage{mathtools}
\usepackage{amsthm}
\usepackage{fancyhdr}
\usepackage{enumitem}
\usepackage{siunitx}
\usepackage{booktabs}
\usepackage[hidelinks]{hyperref}
\usepackage{sectsty}
\usepackage{mathrsfs} % mathscr
\usepackage{tikz}
\usepackage{pgfplots}
\usepackage{multicol}
\usepackage{listings}
% \usepackage{amsart}
\usepackage{fontspec}
\usepackage{soul}


% allow H option of figure
\usepackage{float}

% math font (libertine)
\usepackage{libertinus-otf}

% braket
\usepackage{braket}

% physics
% \usepackage{physics}

% define latin modern font environment
\newcommand{\lms}{\fontfamily{lmss}\selectfont} % Latin Modern Roman
% \newcommand{\lmss}{\fontfamily{lmss}\selectfont} % Latin Modern Sans
% \newcommand{\lmss}{\fontfamily{lmtt}\selectfont} % Latin Modern Mono

% % change mathcal shape
% \usepackage[mathcal]{eucal}


% define math operators
\newcommand{\FF}{\mathbb{F}}
\newcommand{\RR}{\mathbb{R}}
\newcommand{\NN}{\mathbb{N}}
\newcommand{\ZZ}{\mathbb{Z}}
\newcommand{\QQ}{\mathbb{Q}}
\newcommand{\XX}{\mathbb{Y}}
\newcommand{\CL}{\mathcal{L}}
% \renewcommand{\d}{\mathrm{d}}
\renewcommand*\d{\mathop{}\!\mathrm{d}}
\DeclareMathOperator*{\argmax}{arg\,max}
\DeclareMathOperator*{\argmin}{arg\,min}
\DeclareMathOperator{\im}{im}
\DeclareMathOperator{\id}{id}
\DeclareMathOperator{\erf}{erf}
\renewcommand{\mod}[1]{\ (\mathrm{mod}\ #1)}

% section font style
\sectionfont{\sffamily\Large}
\subsectionfont{\sffamily\normalsize}
\subsubsectionfont{\bf}

% line spreading and break
\hyphenpenalty=5000
\tolerance=20
\setlength{\parindent}{0em}
\setlength\parskip{0.5em}
\allowdisplaybreaks
\linespread{0.9}

% enumerate settings
% no break before enumerate
\setlist[enumerate]{itemsep=2pt,topsep=2pt}

% theorem
% latex theorem
% definition style
\theoremstyle{definition}
\newtheorem{theorem}{Theorem}[subsection]
\newtheorem{axiom}{Axiom}[section]
\newtheorem{definition}{Definition}[section]
\newtheorem{example}{Example}[section]
\newtheorem{question}{Question}[section]
\newtheorem{exercise}{Exercise}[section]
\newtheorem*{exercise*}{Exercise}
\newtheorem{lemma}{Lemma}[section]
\newtheorem{proposition}{Proposition}[section]
\newtheorem{corollary}{Corollary}[section]
\newtheorem*{theorem*}{Theorem}
\newtheorem{problem}{Problem}
% remark style
\theoremstyle{remark}
\newtheorem*{remark}{Remark}
\newtheorem*{solution}{Solution}
\newtheorem*{claim}{Claim}


% paragraph indent
\setlength{\parindent}{0em}
\setlength\parskip{0.5em}

\newcommand\Code{PHY5410 FA22}
\newcommand\Ass{HW08}
\newcommand\name{Haoran Sun}
\newcommand\mail{haoransun@link.cuhk.edu.cn}

\title{{\lms \Code \ \Ass}}
\author{\lms \name \ (\href{mailto:\mail}{\mail})}
\date{\sffamily \today}

\makeatletter
% \let\Title\@title
\let\theauthor\@author
\let\thedate\@date

\fancypagestyle{plain}{%
    \fancyhf{}
    \lhead{\sffamily \Ass}
    \rhead{\sffamily \name}
    \rfoot{\sffamily\thepage}

    % # 页脚自定义
    \fancyfoot[L]{
        \begin{minipage}[c]{0.06\textwidth}
            \includegraphics[height=7.5mm]{logo2.png}
        \end{minipage}
    }
}
\fancypagestyle{title}{%
    \fancyhf{}
    \renewcommand{\headrulewidth}{0pt}
    % \lhead{\Title}
    % \rhead{\theauthor}
    \rfoot{\sffamily\thepage}

    % # 页脚自定义
    \fancyfoot[L]{
        \begin{minipage}[c]{0.06\textwidth}
            \includegraphics[height=7.5mm]{logo2.png}
        \end{minipage}
    }
}
\fancyfootoffset[L]{0.3cm}

% re-define title format
\makeatletter
\renewcommand{\maketitle}{\bgroup\setlength{\parindent}{0pt}
\begin{flushleft}
  \textbf{\Large\@title}

  \@author
\end{flushleft}\egroup
}
\makeatother

\pagestyle{plain}

% lstlisting settings
\lstset{
    basicstyle=\linespread{0.7}\footnotesize,
    breaklines=true,
    basewidth=0.5em
}


\begin{document}
\maketitle
\thispagestyle{title}
% \begin{multicols*}{2}

% \begin{remark}
%     $V_\epsilon(x)$ is used to denote a $\epsilon$-neighborhood
%     \begin{align*}
%         V_\epsilon(x) = B_\epsilon(x)\setminus\{x\}
%     \end{align*}
% \end{remark}


\begin{problem}[11.6]\
\begin{enumerate}[label=(\alph*)]
\item 
Let $\mathbf{B} = \nabla\times\mathbf{A}$ where
\begin{align*}
    \mathbf{B} &= \nabla\times\begin{bmatrix}
        0 \\ Bx \\ 0
    \end{bmatrix} = \begin{bmatrix}
        0 \\ 0 \\ B
    \end{bmatrix}
\end{align*}
Then we have
\begin{align*}
    H &= \frac{1}{2m}p^2 + \frac{e^2B^2}{2mc}x^2\Rightarrow
    \frac{1}{2mE}p^2 + \frac{e^2B^2}{2mc^2E}x^2 = 1
\end{align*}
Consider the classical case: pick a particular $p$ s.t.
the equation intersects the $p_x$-$x$ plane and obtains an ellipse.
Let $p_y=p_z=0$ to simplify calculation.
Hence the area enclosed by $p$ and $x$ equals to 
\begin{align*}
    \pi a b &= \frac{2mcE}{eB}\pi 
\end{align*}
Using the Bohr-Sommerfeld quantization condition, we have
\begin{align*}
    \oint\d\mathbf{x}\ \mathbf{p} &= \left(n+\frac{1}{2}\right)h\\
    \oint\d\mathbf{x}\ \mathbf{p} &= \oint\d x\ p_x = \pi ab\\
    \Rightarrow E &= \left(n+\frac{1}{2}\right)\hbar\frac{eB}{mc}
    = \left(n+\frac{1}{2}\right)\hbar\omega
\end{align*}

\hl{Check} $H = H(p, x) = H(L, \theta)$, $H = \cdots + \mathbf{L}\cdot\mathbf{B}$?


\item Solve the limiting case on $xy$ plane by setting $H=0$.
\begin{align*}
    H &= \frac{1}{2m}\left(p - \frac{e}{c}\mathbf{A}\right)^2 = 0
\end{align*}
Hence we have $p_x=p_z=0$, $p_y=eBx/c$.
Thus
\begin{align*}
    \oint \d\mathbf{x}\ \mathbf{p} &= 
    \oint \d y\ \frac{eB}{c}x = \frac{e}{c}\Psi
    = \left(n+\frac{1}{2}\right)\hbar
    \Rightarrow \Phi = \left(n+\frac{1}{2}\right)\frac{hc}{e}
\end{align*}


\end{enumerate}
\end{problem}


\begin{problem}[12.2] 
Define the Hamiltonian of the system as the following expression
\begin{align*}
    H &= \frac{1}{2m}p^2 + \frac{1}{2}m\omega^2 r^2
\end{align*}
And pick eigenfunctions $\phi_{nlm}=R_{nl}(r)Y_{lm_l}$,
where $m_l$ denotes the orbital quantum number and 
$m_s$ denotes the spin quantum number.
Define $N=l+2n$, then for each $N$, we have
\begin{align*}
    N = 2k\Rightarrow l = 0, 2, \dots, N;
    \quad
    N = 2k+1\Rightarrow l = 1, 3, \dots, N
\end{align*}
For each $l$, we have $m_l=-l,-l+1,\dots, l$,
total $2l+1$ degeneracy.
Hence, for both even and odd $N$, we have overall
degeneracy equals to 
\begin{align*}
    \sum_{l=0,2,\dots,N}(2l+1) &= \frac{1}{2}(N+1)(N+2)\\
    \sum_{l=1,3,\dots,N}(2l+1) &= \frac{1}{2}(N+1)(N+2)
\end{align*}
It the particle has spin $s$, the degeneracy becomes to
$(N+1)(N+2)(2s+1)/2$.

In this case, we have $H_2$ equals to
\begin{align*}
    H_2 &= \frac{\omega^2}{2mc^2}\mathbf{S}\cdot\mathbf{L}
\end{align*}
Let $\mathbf{J}=\mathbf{L}+\mathbf{S}$, and pick eigenvector $\ket{j,j_z,l}$, then
\begin{align*}
    H_2\ket{j,j_z,l} &= 
    \frac{\hbar^2}{2}[j(j+1)-s(s+1)-l(l+1)]\ket{j,j_z,l}
\end{align*}
Hence, the spectrum will have the form of
\begin{align*}
    E_{N,j,l} &= 
    \left(N+\frac{3}{2}\right)\hbar\omega 
    + \frac{\hbar^2\omega^2}{2mc^2}[j(j+1)-s(s+1)-l(l+1)]
\end{align*}
Consider the case of spin-$1/2$ particles, if $l>0$
\begin{align*}
    j(j+1)-s(s+1)-l(l+1) &= \begin{cases}
        l & j = l + 1/2\\
        -l-1 & j = l - 1/2
    \end{cases}
\end{align*}
For a fixed $N$, each $l$ will be split into 2 lines
with $j=l+1/2$ and $j=l-1/2$.
\begin{enumerate}[label=(\alph*)]
    \item $l=0$, $E_{N,j,0}$ has degeneracy equals to $2$ corresponds 
    to $j_z=\pm 1/2$ (note that $j=1/2$ when $l=0$).
    \item $l>0$, $E_{N,j,l}$ has degeneracy $2j+1$ corresponds to different $j_z$.
\end{enumerate}
Therefore we have energy described by $N,j,l$ and the degeneracy equals to $2j+1$.


\end{problem}



\begin{problem}[16.2]
Let 
\begin{align*}
    H_0 &= \frac{1}{2m}p^2 + \frac{1}{2}m\omega^2 x,\quad
    H_1 = -eEx
\end{align*}
Let $\ket{n}$ denote the original solutions
and $\ket{n'}$ denote the perturbed solutions.
Then we can verify that
\begin{align*}
    \braket{x-\frac{eE}{m\omega^2}|n} &= 
    \braket{x|n'}
\end{align*}
According to the sudden approximation, the transition probability
of $\psi\rightarrow m'$ would be $|\braket{m'|\psi}|^2$.
Note that
\begin{align*}
    \braket{n'|0} &= 
    \frac{1}{\sqrt{n!}}\braket{(a'_+)^n 0'|0}
\end{align*}
and
\begin{align*}
    a'_+ &= \frac{1}{\sqrt{2m\omega\hbar}}
    \left[m\omega\left(x - \frac{eE}{m\omega^2}\right) - ip\right]
    = a_+ - \frac{1}{\sqrt{2m\omega\hbar}}\frac{eE}{\omega}
    = a_+ - b
\end{align*}
Hence
\begin{align*}
    \braket{n'|0} 
    &= \frac{1}{\sqrt{n!}}\braket{(a_+')^n0'|0}
    = \frac{1}{\sqrt{n!}}\braket{0'|(a'_-)^n0}
    = \frac{1}{\sqrt{n!}}\braket{0'|(-b)^n 0}
    \frac{(-b)^n}{\sqrt{n!}}\braket{0'|0}\\
    &= 
    \frac{(-b)^n}{\sqrt{n!}}A^2
    \int_{-\infty}^\infty\d x\
    \exp\left(-\frac{m\omega}{2\hbar}x^2\right)
    \exp\left[-\frac{m\omega}{2\hbar}(x-eE/m\omega^2)^2\right]\\
    &= \frac{(-b)^n}{\sqrt{n!}}A^2
    \int_{-\infty}^\infty\d x\
    \exp\left[-\left(x-\frac{eE}{2m\omega^2}\right)\right]
    \exp\left(-\frac{e^2E^2}{2m\omega^3 \hbar} \right)\\
    &= \frac{(-b)^n}{\sqrt{n!}}
    \exp\left(-\frac{e^2E^2}{2m\omega^3 \hbar} \right)
\end{align*}

Then $P_{0\rightarrow n'}$ would be
\begin{align*}
    P_{0\rightarrow n'} &= |\braket{n'|0}|^2
    = \frac{b^{2n}}{n!}\exp\frac{2m\omega^3\hbar}{e^2E^2}
    = \frac{1}{n!}\left(\frac{e^2E^2}{2m\omega^3\hbar}\right)^n
    \exp\left(-\frac{e^2E^2}{2m\omega^3\hbar}\right)
\end{align*}
Easy to verify that $\sum_{n'} P_{0\rightarrow n'}=1$.

\end{problem}



% \end{multicols*}
\end{document}

