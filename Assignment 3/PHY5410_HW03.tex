\documentclass[twoside,11pt]{article}
\usepackage[left=1in, right=1in, top=1in, bottom=1in]{geometry}
\usepackage{amsmath}
\usepackage{amssymb}
\usepackage{amsfonts}
\usepackage{mathtools}
\usepackage{amsthm}
\usepackage{fancyhdr}
\usepackage{enumitem}
\usepackage{siunitx}
\usepackage{booktabs}
\usepackage[hidelinks]{hyperref}
\usepackage{sectsty}
\usepackage{mathrsfs} % mathscr
\usepackage{tikz}
\usepackage{pgfplots}
\usepackage{multicol}
\usepackage{listings}
% \usepackage{amsart}
\usepackage{fontspec}


% allow H option of figure
\usepackage{float}

% math font (libertine)
\usepackage{libertinus-otf}

% braket
\usepackage{braket}

% define latin modern font environment
\newcommand{\lms}{\fontfamily{lmss}\selectfont} % Latin Modern Roman
% \newcommand{\lmss}{\fontfamily{lmss}\selectfont} % Latin Modern Sans
% \newcommand{\lmss}{\fontfamily{lmtt}\selectfont} % Latin Modern Mono

% % change mathcal shape
% \usepackage[mathcal]{eucal}


% define math operators
\newcommand{\FF}{\mathbb{F}}
\newcommand{\RR}{\mathbb{R}}
\newcommand{\NN}{\mathbb{N}}
\newcommand{\ZZ}{\mathbb{Z}}
\newcommand{\QQ}{\mathbb{Q}}
\newcommand{\XX}{\mathbb{Y}}
\newcommand{\CL}{\mathcal{L}}
% \renewcommand{\d}{\mathrm{d}}
\renewcommand*\d{\mathop{}\!\mathrm{d}}
\DeclareMathOperator*{\argmax}{arg\,max}
\DeclareMathOperator*{\argmin}{arg\,min}
\DeclareMathOperator{\im}{im}
\DeclareMathOperator{\id}{id}
\renewcommand{\mod}[1]{\ (\mathrm{mod}\ #1)}

% section font style
\sectionfont{\sffamily\Large}
\subsectionfont{\sffamily\normalsize}
\subsubsectionfont{\bf}

% line spreading and break
\hyphenpenalty=5000
\tolerance=20
\setlength{\parindent}{0em}
\setlength\parskip{0.5em}
\allowdisplaybreaks
\linespread{0.9}

% theorem
% latex theorem
% definition style
\theoremstyle{definition}
\newtheorem{theorem}{Theorem}[subsection]
\newtheorem{axiom}{Axiom}[section]
\newtheorem{definition}{Definition}[section]
\newtheorem{example}{Example}[section]
\newtheorem{question}{Question}[section]
\newtheorem{exercise}{Exercise}[section]
\newtheorem*{exercise*}{Exercise}
\newtheorem{lemma}{Lemma}[section]
\newtheorem{proposition}{Proposition}[section]
\newtheorem{corollary}{Corollary}[section]
\newtheorem*{theorem*}{Theorem}
\newtheorem{problem}{Problem}
% remark style
\theoremstyle{remark}
\newtheorem*{remark}{Remark}
\newtheorem*{solution}{Solution}
\newtheorem*{claim}{Claim}


% paragraph indent
\setlength{\parindent}{0em}
\setlength\parskip{0.5em}

\newcommand\Code{PHY5410 FA22}
\newcommand\Ass{HW03}
\newcommand\name{Haoran Sun}
\newcommand\mail{haoransun@link.cuhk.edu.cn}

\title{{\lms \Code \ \Ass}}
\author{\lms \name \ (\href{mailto:\mail}{\mail})}
\date{\sffamily \today}

\makeatletter
% \let\Title\@title
\let\theauthor\@author
\let\thedate\@date

\fancypagestyle{plain}{%
    \fancyhf{}
    \lhead{\sffamily \Ass}
    \rhead{\sffamily \name}
    \rfoot{\sffamily\thepage}

    % # 页脚自定义
    \fancyfoot[L]{
        \begin{minipage}[c]{0.06\textwidth}
            \includegraphics[height=7.5mm]{logo2.png}
        \end{minipage}
    }
}
\fancypagestyle{title}{%
    \fancyhf{}
    \renewcommand{\headrulewidth}{0pt}
    % \lhead{\Title}
    % \rhead{\theauthor}
    \rfoot{\sffamily\thepage}

    % # 页脚自定义
    \fancyfoot[L]{
        \begin{minipage}[c]{0.06\textwidth}
            \includegraphics[height=7.5mm]{logo2.png}
        \end{minipage}
    }
}
\fancyfootoffset[L]{0.3cm}

% re-define title format
\makeatletter
\renewcommand{\maketitle}{\bgroup\setlength{\parindent}{0pt}
\begin{flushleft}
  \textbf{\Large\@title}

  \@author
\end{flushleft}\egroup
}
\makeatother

\pagestyle{plain}

% lstlisting settings
\lstset{
    basicstyle=\linespread{0.7}\footnotesize,
    breaklines=true,
    basewidth=0.5em
}


\begin{document}
\maketitle
\thispagestyle{title}
% \begin{multicols*}{2}

% \begin{remark}
%     $V_\epsilon(x)$ is used to denote a $\epsilon$-neighborhood
%     \begin{align*}
%         V_\epsilon(x) = B_\epsilon(x)\setminus\{x\}
%     \end{align*}
% \end{remark}

\begin{problem}[5.1]\
\begin{enumerate}[label=(\alph*)]
    \item We can show that $\braket{L_x}=0$ algebraically.
    Similarly, we can also show that $\braket{L_y}=0$.
    Then, $L_{\pm}$ vanish automatically since $L_{\pm} = L_x \pm i L_y$.
    \begin{align*}
        i\hbar\braket{Y_{l,l}|L_xY_{l,l}}
        &= \braket{Y_{l,l}|[L_y, L_z]Y_{l,l}}
        = \braket{Y_{l,l}|(L_yL_z-L_zL_y)Y_{l,l}} \\
        &= l\braket{Y_{l,l}|L_y Y_{l,l}} - \braket{Y_{l,l}|L_zL_yY_{l,l}}
        = l\braket{Y_{l,l}|L_y Y_{l,l}} - \braket{Y_{l,l}|L_zL_yY_{l,l}}\\
        &= l\braket{Y_{l,l}|L_yY_{l,l}} - \braket{L_yL_zY_{l,l}|Y_{l,l}}
        = l\braket{Y_{l,l}|L_yY_{l,l}} - l\braket{Y_{l,l}|L_yY_{l,l}}\\
        &= 0\Rightarrow \braket{L_x}=0
    \end{align*}

    \item By (a) we know that $\braket{L_x} = \braket{L_y}=0$, then $\Delta L_x^2 = \braket{L_x^2}$ 
    and $\Delta L_y^2 = \braket{L_y^2}$.
    Using the fact that $\braket{L_z^2} = l^2\hbar^2$ and $\braket{\mathbf{L}^2}=(l^2+l)\hbar^2$, we can derive that
    $\Delta L_x^2 + \Delta L_y^2=l\hbar^2$.
    \begin{align*}
        \mathbf{L}^2 &= L_x^2 + L_y^2 + L_z^2\\
        \Rightarrow
        \braket{\mathbf{L}^2} &= \braket{L_x^2} + \braket{L_y^2} + \braket{L_z^2} = l(l+1)\hbar^2  \\
        \Rightarrow l\hbar^2 &= \Delta L_x^2 + \Delta L_y^2
    \end{align*}
    Also, according to the uncertainty relationship $\Delta A\Delta B\geq |\braket{[A,B]}|/2$, we have
    \begin{align*}
        \Delta L_x\Delta L_y \geq \frac{1}{2} \braket{i\hbar L_z} = \frac{1}{2}\hbar^2l
    \end{align*}
    Note that the following set of equations could only have a unique solution
    $\Delta L_x=\Delta L_y=\hbar\sqrt{l/2}$.
    Apparently, in the state $Y_{l,l}$, we have $\Delta L_z = 0$.
    \begin{align*}
        \begin{cases}
            \Delta L_x^2 + \Delta L_y^2 = l\hbar^2\\
            \Delta L_x\Delta L_y \geq \hbar^2 l/2
        \end{cases}
    \end{align*}

    \item From (b) we know that $\Delta L_x^2 + \Delta L_y^2 = \hbar^2l(l-1) - \hbar^2m^2$.
    Hence the expression $\Delta L_x^2 + \Delta L_y^2$ takes its minimum when $m^2$ takes
    its maximum, i.e., $m=\pm l$.


\end{enumerate}
\end{problem}


\begin{problem}[5.6]\
    \begin{enumerate}[label=(\alph*)]
        \item If we write $Y_{1,1}$ under the cartesian coordinate, we would get
        \begin{align*}
            Y_{1,1} = Y_{1,1}(x,y,z),\
            L_z Y_{1, 1}(x, y, z) = \hbar^2 Y_{1, 1}(x, y, z)
        \end{align*}
        Suppose we perform a rotational transformation $(x, y, z)\rightarrow(-z, y, x)$, 
        then $Y_{1, 1}(-z, y, x)$ would directly be the eigenfunction of $L_z$ with eigenvalue
        $\hbar^2l$.
        Since $\mathbf{L}^2$ is rotational invariant, we still have $\mathbf{L}^2Y_{11}(-z, y, x) =\hbar^2l(l+1)
        Y_{11}(-z, y, x)$.
        \begin{align*}
            L_xY_{1, 1}(-z, y, x) 
            &= (yp_z - zp_y) Y_{1,1}(-z, y, x)
            = [(-z)p_y - yp_{(-z)}] Y_{1,1}(-z,y,x)\\
            &= [x'p_y'  - y' p_{x'}] Y_{1,1}(x',y',x)
            = \hbar^2l Y_{1,1}(x', y', x)\\
            &= \hbar^2 l Y_{1,1}(-z, y, x)
        \end{align*}
        
        \item Using the algebraic properties of angular momentum to solve this problem.
        \begin{remark}
            For convenience, I would like to drop all $\hbar$ terms in the equation, i.e.
            I would written $L_z Y_{1,1} = Y_{1,1}$ instead of $L_z Y_{1,1}=\hbar^2 Y_{1,1}$.
            For notations, let $X_{1,1}$ denote the eigenfunction of $L_x$ with
            $L_x X_{1,1} = X_{1,1}$ and $\mathbf{L}^2 X_{1,1} = 2 X_{1,1}$.
        \end{remark}
        Represent the angular momentum operator under the matrix algebra: 
        let $Y\in\RR^n$ and $L_x, L_y, L_z\in\RR^{n\times n}$.
        \begin{align*}
            aY_{1,-1} + bY_{1,0} + cY_{0,1}\mapsto \begin{bmatrix}
                a\\ b\\ c
            \end{bmatrix}
        \end{align*}
        According to the expression of ladder operator $L_+=L_x+iL_y$ and 
        $L_-=L_x-iL_y$, we have
        \begin{align*}
            L_+ &= L_x + i L_y = \sqrt{2}\begin{bmatrix}
                0 & 0 & 0 \\
                1 & 0 & 0 \\
                0 & 1 & 0
            \end{bmatrix}\\
            L_- &= L_x - i L_y = \sqrt{2}\begin{bmatrix}
                0 & 1 & 0 \\
                0 & 0 & 1 \\
                0 & 0 & 0
            \end{bmatrix}
        \end{align*}
        where the $\sqrt{2}$ term comes from the normalization rule.
        Therefore we can solve $L_x$ and $L_y$ accordingly.
        \begin{align*}
            L_x = \frac{1}{\sqrt{2}}\begin{bmatrix}
                0 & 1 & 0\\
                1 & 0 & 1\\
                0 & 1 & 0
            \end{bmatrix}, 
            L_y = \frac{1}{\sqrt{2}}\begin{bmatrix}
                0 & i & 0\\
                -i & 0 & i\\
                0 & -i & 0
            \end{bmatrix}
        \end{align*}
        By some calculations, we can show that the characteristic equation of
        $L_x$ is $\lambda^3 - \lambda=0$, which means $L_x$ have three distinct
        eigenvalues $-1$, $0$, and $1$.
        Then, the eigenvector corresponding to $\lambda=1$ could be explicitly determined by
        finding $\ker (L_x - I)$. Due to the normalization condition, the vector can
        be $v_1 = \begin{bmatrix} 1/2 & 1/\sqrt{2} & 1/2\end{bmatrix}^T$.
        \begin{align*}
            v_1&\in\ker (L_x-I) = \ker\begin{bmatrix}
                -1 & 1/\sqrt{2} & 0\\
                1/\sqrt{2} & -1 & 1/\sqrt{2}\\
                0 & 1/\sqrt{2} & -1
            \end{bmatrix}  \text{ and } \|v_1\| = 1
            \Rightarrow v_1 = \begin{bmatrix}
                1/2 \\ 1/\sqrt{2} \\ 1/2
            \end{bmatrix} \text{ a solution}
        \end{align*}
        Therefore $X_{1,1} = Y_{1,-1}/2 + Y_{1,0}/\sqrt{2} + Y_{1,1}/2$ is a normalized eigenfunction
        of $L_x$ with $L_x X_{1,1} = \hbar^2 X_{1,1}$.
    \end{enumerate}
\end{problem}


\begin{problem}[6.2]
    \begin{claim}
        The operator $\mathcal{L} := e^x (\d/\d x) e^{-x}= \d/\d x - 1$, 
        and $L_r(x) = \mathcal{L}^r x^r$.
    \end{claim}
    \begin{proof}
        $\forall f$
        \begin{align*}
            \CL = e^x\frac{\d }{\d x}e^{-x} = e^x\left(
                e^{-x}\frac{\d }{\d x}-e^{-x}\right) = 
            \frac{\d }{\d x} - 1
        \end{align*}
        also 
        \begin{align*}
            e^x\frac{\d^2}{\d x^2}e^{-x} = e^x\frac{\d}{\d x}\left(
                e^{-x}\frac{\d}{\d x} - e^{-x}
            \right)
            = e^x\left(
                e^{-x}\frac{\d^2}{\d x^2} - 2e^{-x}\frac{\d}{\d x} + e^{-x}
            \right)
            = \left(\frac{\d}{\d x} - 1\right)^2 = \CL^2
        \end{align*}
        Then we can prove by induction that $\CL^n = e^x(\d/\d x)^n e^{-x}$.
    \end{proof}
    \begin{enumerate}[label=(\alph*)]
        \item Using the fact that $L_r(x) = \CL^n(x^n$
        \begin{align*}
            L_r(x) = \left(\frac{\d}{\d x} - 1\right)^r x^r = 
            \sum_{k=0}^r (-1)^k\begin{pmatrix}
                r\\k
            \end{pmatrix}\left(\frac{\d}{\d x}\right)^{r-k} x^r
            = \sum_{k=0}^r (-1)^k\begin{pmatrix}
                r\\k
            \end{pmatrix}\frac{r!}{k!}x^k
        \end{align*}
        Therefore
        \begin{align*}
            L^s_r(x)
            &= \frac{\d^s}{\d x^s} L_r(x)
            = \sum_{k=s}^r (-1)^k\begin{pmatrix}
                r\\k
            \end{pmatrix} \frac{r!}{k!}\frac{k!}{(k-s)!}x^{k-s}\\
            &= \sum_{k=s}^r (-1)^k \frac{[r!]^2}{(r-k)!(k-s)!}x^{k-s}
            = \sum_{u=0}^{r-s}(-1)^{u+s}\frac{[r!]^2}{(u+s)!(r-u-s)!}x^u
        \end{align*}


        \item Since $[\d/\d x, 1]=[\d/\d x, \d/\d x]=0$, then we have
        \begin{align*}
            L_{r+m}^m = \left(\frac{\d}{\d x}\right)^m \left(
                \frac{\d }{\d x}-1
            \right)^{r+m}  x^{r+m}
            =  \left(
                \frac{\d }{\d x}-1
            \right)^{r+m} \left(\frac{\d}{\d x}\right)^m x^{r+m}
            = \left(
                \frac{\d }{\d x}-1
            \right)^{r+m} \frac{(r+m)!}{r!} x^r
        \end{align*}
        Then
        \begin{align*}
            \sum_{n=0}^\infty \frac{t^r}{(r+m)!}L_{r+m}^m(x)
            &= \sum_{r=0}^\infty \left(\frac{\d }{\d x} - 1\right)^{r+m}\frac{(r+m)!}{r!} x^r\frac{1}{(r+m)!}t^r\\
            &= \left(\frac{\d }{\d x} - 1\right)^m \sum_{r=0}^\infty\left(\frac{\d }{\d x} - 1\right)^r t^rx^r\frac{1}{r!}\\
            &= \left(\frac{\d }{\d x} - 1\right)^m \sum_{r=0}^\infty\frac{t^r}{r!}\left(\frac{\d }{\d x} - 1\right)^r x^r\\
            &= \left(\frac{\d }{\d x} - 1\right)^m \sum_{r=0}^\infty \frac{t^r}{r!} L_r(x) \\
            &= \left(\frac{\d }{\d x} - 1\right)^m \frac{1}{1-t}\exp\left(-x\frac{t}{1-t}\right)\\
            &= \frac{(-1)^m}{(1-t)^{m+1}}\exp\left(-x\frac{t}{1-t}\right)
        \end{align*}

    \end{enumerate}
\end{problem}








% \end{multicols*}
\end{document}

