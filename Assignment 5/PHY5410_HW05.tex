\documentclass[twoside,11pt]{article}
\usepackage[left=1in, right=1in, top=1in, bottom=1in]{geometry}
\usepackage{amsmath}
\usepackage{amssymb}
\usepackage{amsfonts}
\usepackage{mathtools}
\usepackage{amsthm}
\usepackage{fancyhdr}
\usepackage{enumitem}
\usepackage{siunitx}
\usepackage{booktabs}
\usepackage[hidelinks]{hyperref}
\usepackage{sectsty}
\usepackage{mathrsfs} % mathscr
\usepackage{tikz}
\usepackage{pgfplots}
\usepackage{multicol}
\usepackage{listings}
% \usepackage{amsart}
\usepackage{fontspec}


% allow H option of figure
\usepackage{float}

% math font (libertine)
\usepackage{libertinus-otf}

% braket
\usepackage{braket}

% define latin modern font environment
\newcommand{\lms}{\fontfamily{lmss}\selectfont} % Latin Modern Roman
% \newcommand{\lmss}{\fontfamily{lmss}\selectfont} % Latin Modern Sans
% \newcommand{\lmss}{\fontfamily{lmtt}\selectfont} % Latin Modern Mono

% % change mathcal shape
% \usepackage[mathcal]{eucal}


% define math operators
\newcommand{\FF}{\mathbb{F}}
\newcommand{\RR}{\mathbb{R}}
\newcommand{\NN}{\mathbb{N}}
\newcommand{\ZZ}{\mathbb{Z}}
\newcommand{\QQ}{\mathbb{Q}}
\newcommand{\XX}{\mathbb{Y}}
\newcommand{\CL}{\mathcal{L}}
% \renewcommand{\d}{\mathrm{d}}
\renewcommand*\d{\mathop{}\!\mathrm{d}}
\DeclareMathOperator*{\argmax}{arg\,max}
\DeclareMathOperator*{\argmin}{arg\,min}
\DeclareMathOperator{\im}{im}
\DeclareMathOperator{\id}{id}
\renewcommand{\mod}[1]{\ (\mathrm{mod}\ #1)}

% section font style
\sectionfont{\sffamily\Large}
\subsectionfont{\sffamily\normalsize}
\subsubsectionfont{\bf}

% line spreading and break
\hyphenpenalty=5000
\tolerance=20
\setlength{\parindent}{0em}
\setlength\parskip{0.5em}
\allowdisplaybreaks
\linespread{0.9}

% theorem
% latex theorem
% definition style
\theoremstyle{definition}
\newtheorem{theorem}{Theorem}[subsection]
\newtheorem{axiom}{Axiom}[section]
\newtheorem{definition}{Definition}[section]
\newtheorem{example}{Example}[section]
\newtheorem{question}{Question}[section]
\newtheorem{exercise}{Exercise}[section]
\newtheorem*{exercise*}{Exercise}
\newtheorem{lemma}{Lemma}[section]
\newtheorem{proposition}{Proposition}[section]
\newtheorem{corollary}{Corollary}[section]
\newtheorem*{theorem*}{Theorem}
\newtheorem{problem}{Problem}
% remark style
\theoremstyle{remark}
\newtheorem*{remark}{Remark}
\newtheorem*{solution}{Solution}
\newtheorem*{claim}{Claim}


% paragraph indent
\setlength{\parindent}{0em}
\setlength\parskip{0.5em}

\newcommand\Code{PHY5410 FA22}
\newcommand\Ass{HW05}
\newcommand\name{Haoran Sun}
\newcommand\mail{haoransun@link.cuhk.edu.cn}

\title{{\lms \Code \ \Ass}}
\author{\lms \name \ (\href{mailto:\mail}{\mail})}
\date{\sffamily \today}

\makeatletter
% \let\Title\@title
\let\theauthor\@author
\let\thedate\@date

\fancypagestyle{plain}{%
    \fancyhf{}
    \lhead{\sffamily \Ass}
    \rhead{\sffamily \name}
    \rfoot{\sffamily\thepage}

    % # 页脚自定义
    \fancyfoot[L]{
        \begin{minipage}[c]{0.06\textwidth}
            \includegraphics[height=7.5mm]{logo2.png}
        \end{minipage}
    }
}
\fancypagestyle{title}{%
    \fancyhf{}
    \renewcommand{\headrulewidth}{0pt}
    % \lhead{\Title}
    % \rhead{\theauthor}
    \rfoot{\sffamily\thepage}

    % # 页脚自定义
    \fancyfoot[L]{
        \begin{minipage}[c]{0.06\textwidth}
            \includegraphics[height=7.5mm]{logo2.png}
        \end{minipage}
    }
}
\fancyfootoffset[L]{0.3cm}

% re-define title format
\makeatletter
\renewcommand{\maketitle}{\bgroup\setlength{\parindent}{0pt}
\begin{flushleft}
  \textbf{\Large\@title}

  \@author
\end{flushleft}\egroup
}
\makeatother

\pagestyle{plain}

% lstlisting settings
\lstset{
    basicstyle=\linespread{0.7}\footnotesize,
    breaklines=true,
    basewidth=0.5em
}


\begin{document}
\maketitle
\thispagestyle{title}
% \begin{multicols*}{2}

% \begin{remark}
%     $V_\epsilon(x)$ is used to denote a $\epsilon$-neighborhood
%     \begin{align*}
%         V_\epsilon(x) = B_\epsilon(x)\setminus\{x\}
%     \end{align*}
% \end{remark}


\begin{problem}[8.1]
Define operator $U=e^{-iHt/\hbar}$, then we have
\begin{align*}
    A_HB_H - B_HA_H &= (U^\dagger AU)(U^\dagger BU) - 
    (U^\dagger BU)(U^\dagger AU) = 
    U^\dagger AB U - U^\dagger BA U 
    = U^\dagger [A, B] U = C_H
\end{align*}
\end{problem}


\begin{problem}[8.2]
Using the fact that
\begin{align*}
    [x_H, p_H] &= [x, p]_H = i\hbar\\
    [x, H] &= \frac{1}{2m}[x, p^2] = \frac{i}{m}\hbar p\\
    [H, p] &= \frac{m\omega^2}{2}[x^2, p] = i\hbar m\omega^2 x
\end{align*}
we have
\begin{align*}
    \dot{x}_H &= \frac{i}{\hbar}[H_H, x_H]
    = \frac{i}{\hbar}\frac{-i}{m}\hbar p_H = \frac{p_H}{m}\\
    \dot{p}_H &= \frac{i}{\hbar}[H_H, p_H] 
    = \frac{i}{\hbar}i\hbar m\omega^2 x_H = -m\omega^2 x_H
\end{align*}
Then we have a system of differential equations
\begin{align*}
    \frac{\d }{\d t}\begin{bmatrix}
        x_H\\ p_H
    \end{bmatrix}
    &=\begin{bmatrix}
        0 & 1/m\\
        -m\omega^2 & 0
    \end{bmatrix}
    \begin{bmatrix}
        x_H\\ p_H
    \end{bmatrix}
\end{align*}
The solution is
\begin{align*}
    \begin{bmatrix}
        x_H(t)\\ p_H(t)
    \end{bmatrix}
    &= 
    \exp\left(
    \begin{bmatrix}
        0 & 1/m\\
        -m\omega^2 & 0
    \end{bmatrix}t
    \right)
    \begin{bmatrix}
        x_H(0)\\ p_H(0)
    \end{bmatrix}
\end{align*}
Since
\begin{align*}
    \begin{bmatrix}
        0 & 1/m\\
        -m\omega^2 & 0
    \end{bmatrix}t &=
    \frac{1}{2im\omega}
    \begin{bmatrix}
        1 & -1 \\ im\omega & im\omega
    \end{bmatrix}
    \begin{bmatrix}
        i\omega t & \\ & -i\omega t
    \end{bmatrix}
    \begin{bmatrix}
        im\omega & 1\\ -im\omega & 1
    \end{bmatrix}
\end{align*}
we have
\begin{align*}
    \exp\left(\begin{bmatrix}
        0 & 1/m\\ -m\omega^2 & 0
    \end{bmatrix}t \right)
    &= 
    \frac{1}{2im\omega}
    \begin{bmatrix}
        1 & -1 \\ im\omega & im\omega
    \end{bmatrix}
    \begin{bmatrix}
        e^{i\omega t} & \\ & e^{-i\omega t}
    \end{bmatrix}
    \begin{bmatrix}
        im\omega & 1\\ -im\omega & 1
    \end{bmatrix}
    = 
    \begin{bmatrix}
        \cos\omega t & \dfrac{1}{m\omega}\sin\omega t\\
        -m\omega\sin\omega t & \cos\omega t
    \end{bmatrix}
\end{align*}
Therefore
\begin{align*}
    x_H(t) &= \cos\omega t x_H(0) + \dfrac{1}{m\omega}\sin\omega t p_H(0)\\
    p_H(t) &= \cos\omega t p_H(0) - m\omega\sin\omega t x_H(0)
\end{align*}
Using the fact that $a(t)=(m\omega x - ip)/\frac{1}{2m\hbar}$, we have
\begin{align*}
    a_H(t) &= \frac{1}{\sqrt{2m\hbar}}[m\omega x_H(t) - ip_H(t)]
    = \frac{1}{\sqrt{2m\hbar}}[m\omega e^{i\omega}x_H(0) - ie^{-i\omega}p_H(0)]
\end{align*}

\end{problem}



\begin{problem}[8.4]\
\begin{enumerate}[label=Case \arabic*., leftmargin=*]
\item $l=1/2$, we have $m=\pm 1/2$, let 
\begin{align*}
    \alpha\ket{1/2} + \beta\ket{-1/2} \mapsto
    \begin{bmatrix}
        \alpha\\ \beta
    \end{bmatrix}
\end{align*}
then under the matrix representation, we have
\begin{align*}
    L^2 &= \hbar^2 \frac{3}{4}\mathbf{I}\\
    L_z &= \frac{\hbar}{2}\begin{bmatrix}
        1 & 0\\
        0 & -1
    \end{bmatrix}\\
    L_x &= \frac{1}{2}(L_+ + L_-)
    = \frac{\hbar}{2}\left(
        \begin{bmatrix}
            0 & 1 \\ 0 & 0
        \end{bmatrix}
        + \begin{bmatrix}
            0 & 0 \\ 1 & 0
        \end{bmatrix}
    \right) = \frac{\hbar}{2}
    \begin{bmatrix}
        0 & 1\\ 1 & 0
    \end{bmatrix}\\
    L_y &= \frac{1}{2i}(L_+ - L_-) 
    = \frac{\hbar}{2i}\begin{bmatrix}
        0 & 1\\ -1 & 0
    \end{bmatrix}
\end{align*}

\item $l=1$, let
\begin{align*}
    \alpha\ket{1} + \beta\ket{0} + \gamma\ket{-1}\mapsto
    \begin{bmatrix}
        \alpha\\ \beta\\ \gamma
    \end{bmatrix}
\end{align*}
we have
\begin{align*}
    L^2 &= 2\hbar^2\mathbf{I}\\
    L_z &= \frac{\hbar}{2}\begin{bmatrix}
        1 & & \\
        & 0 & \\
        & & -1
    \end{bmatrix}\\
    L_+ &= \hbar\begin{bmatrix}
        & \sqrt[]{2} & \\
        & & \sqrt[]{2}\\
        & & 
    \end{bmatrix}\\
    L_- &= \hbar\begin{bmatrix}
        & & \\
        \sqrt[]{2} & & \\
        & \sqrt[]{2} & 
    \end{bmatrix}\\
    L_x &= \frac{\hbar}{2}\begin{bmatrix}
        0 & \sqrt[]{2} & 0\\
        \sqrt[]{2} & 0 & \sqrt[]{2}\\
        0 & \sqrt[]{2} & 0
    \end{bmatrix}\\
    L_y &= \frac{\hbar}{2i}\begin{bmatrix}
        0 & \sqrt[]{2} & 0\\
        -\sqrt[]{2} & 0 & \sqrt[]{2}\\
        0 & -\sqrt[]{2} & 0
    \end{bmatrix}
\end{align*}


\item  $l=3/2$, let
\begin{align*}
    \alpha\ket{3/2} + \beta\ket{1/2} + \gamma\ket{-1/2} + \delta\ket{-3/2}
    \mapsto
    \begin{bmatrix}
        \alpha\\ \beta\\ \gamma\\ \delta
    \end{bmatrix}
\end{align*}
we have
\begin{align*}
    L^2 &= \frac{15}{4}\hbar^2\mathbf{I}\\
    L_z &= \hbar\begin{bmatrix}
        3/2 & & & \\
        & 1/2 & & \\
        & & -1/2 & \\ 
        & & & -3/2
    \end{bmatrix}\\
    L_+ &= \hbar\begin{bmatrix}
        & \sqrt[]{3} & & \\
        & & 2 & \\
        & & & \sqrt[]{3}\\
        & & & 
    \end{bmatrix}\\
    L_- &= \hbar\begin{bmatrix}
        & & & \\
        \sqrt[]{3} & & & \\
        & 2 & & \\ 
        & & \sqrt[]{3} & 
    \end{bmatrix}\\
    L_x &= \frac{\hbar}{2}\begin{bmatrix}
        0 & \sqrt[]{3} & 0 & 0\\
        \sqrt[]{3} & 0 & 2 & 0\\
        0 & 2 & 0 & \sqrt[]{3}\\
        0 & 0 & \sqrt[]{3} & 0
    \end{bmatrix}\\
    L_y &= \frac{\hbar}{2i}\begin{bmatrix}
        0 & \sqrt[]{3} & 0 & 0\\
        -\sqrt[]{3} & 0 & 2 & 0\\
        0 & -2 & 0 & \sqrt[]{3}\\
        0 & 0 & -\sqrt[]{3} & 0
    \end{bmatrix}
\end{align*}

\item $l=2$, let
\begin{align*}
    \alpha\ket{2} + \beta\ket{1} + \gamma\ket{0} + \delta\ket{-1} + \epsilon\ket{-2}
    \mapsto
    \begin{bmatrix}
        \alpha\\ \beta\\ \gamma\\ \delta\\ \epsilon
    \end{bmatrix}
\end{align*}
we have 
\begin{align*}
    L^2 &= 6\hbar^2 \mathbf{I}\\ 
    L_z &= \hbar\begin{bmatrix}
        2 & & & & \\
        & 1 & & & \\
        & & 0 & & \\
        & & & -1 & \\
        & & & & -2
    \end{bmatrix}\\
    L_+ &= \hbar\begin{bmatrix}
        & 2 & & & \\
        & & \sqrt[]{6} & & \\
        & & & \sqrt[]{6} & \\
        & & & & 2\\
        & & & & 
    \end{bmatrix}\\
    L_- &= \hbar\begin{bmatrix}
        & & & & \\
        2 & & & & \\
        & \sqrt[]{6} & & & \\
        & & \sqrt[]{6} & & \\
        & & & 2 & 
    \end{bmatrix}\\
    L_x &= \frac{\hbar}{2}\begin{bmatrix}
        0 & 2 & 0 & 0 & 0 \\
        2 & 0 & \sqrt[]{6} & 0 & 0 \\
        0 & \sqrt[]{6} & 0 & \sqrt[]{6} & 0\\
        0 & 0 & \sqrt[]{6} & 0 & 2\\
        0 & 0 & 0 & 2 & 0
    \end{bmatrix}\\
    L_y &= \frac{\hbar}{2i}\begin{bmatrix}
        0 & 2 & 0 & 0 & 0 \\
        -2 & 0 & \sqrt[]{6} & 0 & 0 \\
        0 & -\sqrt[]{6} & 0 & \sqrt[]{6} & 0\\
        0 & 0 & -\sqrt[]{6} & 0 & 2\\
        0 & 0 & 0 & -2 & 0
    \end{bmatrix}
\end{align*}


\end{enumerate}
\end{problem}



\begin{problem}[8.5]\
\begin{enumerate}[label=(\alph*)]
\item Let $\mathbf{p}_i$ denotes the momentum operator of the $i$th particle
\begin{align*}
    \mathbf{p}_i &= \begin{bmatrix}
        p_{ix}\\ p_{iy}\\ p_{iz}
    \end{bmatrix}
\end{align*}
Then easy to derive that
\begin{align*}
    [H, \mathbf{P}] &= 
    \left(\sum_{ij}\frac{1}{2}V(|\mathbf{x}_i-\mathbf{x}_j|)\right)
    \left(\sum_n \mathbf{p}_n\right) - 
    \left(\sum_n \mathbf{p}_n\right)
    \left(\sum_{ij}\frac{1}{2}V(|\mathbf{x}_i-\mathbf{x}_j|)\right)\\
    &=
    \sum_{nj}
    V(r_{nj})
    \mathbf{p}_n
    -
    \sum_{nj}
    \mathbf{p}_n
    V(r_{nj})
\end{align*}
By the commutation relationship where
\begin{align*}
    p_{i\alpha}\frac{1}{r_{ij}} &= i\hbar\frac{x_{i\alpha} - x_{j\alpha}}{r_{ij}^3}
    + \frac{1}{r_{ij}}p_{i\alpha}
\end{align*}
where $\alpha=x, y, z$.
Using the fact that $V(r_{ij})= \sum_k 1/r^k_{ij}$,
we have
\begin{align*}
    \sum_{nj}p_{n\alpha}\frac{1}{r_{nj}} &= 
    \sum_{n<j} p_{n\alpha}\frac{1}{r_{nj}} + p_{j\alpha} \frac{1}{r_{nj}}\\
    &= \sum_{n<j}
    i\hbar k \frac{x_{n\alpha} - x_{j\alpha}}{r_{nj}^{k+2}} + \frac{1}{r_{nj}}p_{n\alpha}
    +
    i\hbar k \frac{x_{j\alpha} - x_{n\alpha}}{r_{nj}^{k+2}} + \frac{1}{r_{nj}}p_{j\alpha}\\
    &= 
    \sum_{n<j}
    \frac{1}{r_{nj}}p_{n\alpha} + \frac{1}{r_{nj}}p_{j\alpha}\\
    &= \sum_{nj} \frac{1}{r_{nj}}p_{n\alpha}
\end{align*}
Hence $\sum_{nj}
V(r_{nj})
\mathbf{p}_n
-
\sum_{nj}
\mathbf{p}_n
V(r_{nj})=0$ and $[H,\mathbf{P}]$.

The proof of $[H,\mathbf{L}]=0$ begins with the following claims
\begin{claim}
Let $L_{n\alpha}$ denotes the $\alpha$ component ($\alpha=x,y,z$)
of the angular momentum operator of $n$th particle in the system,
$L_{n\alpha} = (\mathbf{x}_n\times\mathbf{p}_n)_\alpha$.
Then
\begin{align*}
    \left[
        \sum_n \mathbf{p}_n^2,
        \sum_n L_{n\alpha}
    \right] = 0
\end{align*}
\end{claim}
\begin{proof}
    Note that
    \begin{align*}
        \left[
        \sum_n \mathbf{p}_n^2,
        \sum_n L_{n\alpha}
    \right] &=
    \left(\sum_n\mathbf{p}_n^2\right)
    \left(\sum_n \epsilon_{\alpha\beta\gamma}x_{n\beta}p_{n\gamma}\right)
    -\left(\sum_n \epsilon_{\alpha\beta\gamma}x_{n\beta}p_{n\gamma}\right)
    \left(\sum_n\mathbf{p}_n^2\right)\\
    &= \sum_n\epsilon_{\alpha\beta\gamma} \mathbf{p}_n^2x_{n\beta}p_{n\gamma}
    -\sum_n\epsilon_{\alpha\beta\gamma}x_{n\beta}p_{n\gamma}\mathbf{p}_n^2\\
    &= \sum_n\epsilon_{\alpha\beta\gamma} (x_{n\beta}\mathbf{p}_n^2 - 2i\hbar p_{n\beta})p_{n\gamma}
    -\sum_n\epsilon_{\alpha\beta\gamma}x_{n\beta}p_{n\gamma}\mathbf{p}_n^2\\
    &= -2i\hbar\sum_n\epsilon_{\alpha\beta\gamma}p_{n\beta}p_{n\gamma} = 0\qedhere
    \end{align*}
\end{proof}
\begin{claim}
\begin{align*}
    \left[\sum_{ij}\frac{1}{r^k_{ij}}, \sum_n L_{n\alpha}\right] = 0
\end{align*}
\end{claim}
\begin{proof}
Note that
\begin{align*}    
    \left[\sum_{ij}\frac{1}{r^k_{ij}}, \sum_n L_{n\alpha}\right] &=
    \left(\sum_{ij}\frac{1}{r^k_{ij}}\right)
    \left(\sum_n\epsilon_{\alpha\beta\gamma}x_{n\beta}p_{n\gamma}\right)
    -\left(\sum_n\epsilon_{\alpha\beta\gamma}x_{n\beta}p_{n\gamma}\right)
    \left(\sum_{ij}\frac{1}{r^k_{ij}}\right)\\
    &= 2\sum_{nj}\epsilon_{\alpha\beta\gamma}\frac{1}{r^k_{nj}}x_{n\beta}p_{n\gamma}
    - 2\sum_{nj}\epsilon_{\alpha\beta\gamma}x_{n\beta}p_{n\gamma}\frac{1}{r^k_{nj}}\\
    &= 2\sum_{nj}\epsilon_{\alpha\beta\gamma}\frac{1}{r^k_{nj}}x_{n\beta}p_{n\gamma}
    - 2\sum_{nj}\epsilon_{\alpha\beta\gamma}x_{n\beta}\left(
        i\hbar k \frac{x_{n\gamma} - x_{j\gamma}}{r_{nj}^{k+2}} + \frac{1}{r^k_{nj}}p_{n\gamma}
    \right)\\
    &= -2i\hbar k \sum_{nj}\epsilon_{\alpha\beta\gamma}x_{n\beta}\frac{x_{n\gamma}-x_{j\gamma}}{r_{nj}^{k+2}}
\end{align*}
Since
\begin{align*}
    \sum_{nj}\epsilon_{\alpha\beta\gamma}x_{n\beta}x_{n\gamma} &= 0\\
    \sum_{nj}\epsilon_{\alpha\beta\gamma}x_{n\beta}x_{j\gamma} &= 
    \sum_{n<j}\epsilon_{\alpha\beta\gamma}(x_{n\beta}x_{j\gamma} + x_{j\beta}x_{n\gamma})\\
    &=
    \sum_{n<j}\epsilon_{\alpha\beta\gamma}(x_{n\beta}x_{j\gamma} - x_{j\gamma}x_{n\beta})
    = 0
\end{align*}
Therefore we have
\begin{align*}
    \left[\sum_{nj}\frac{1}{r^k_{nj}}, \sum_n L_{n\alpha}\right] &= 0 \qedhere
\end{align*}
\end{proof}
According to these two claims, we have $[H,\mathbf{L}] = 0$.


\item Since the laplacian operator is invariant to Euclidean transformations,
and $|\mathbf{x}_i-\mathbf{x}_j|$ also invariant to rotations and translations,
we can verify that the following operators are invariant to rotations and translations.
\begin{align*}
    \sum_n \mathbf{p}_n^2,\
    \sum_{ij} V(|\mathbf{x}_i-\mathbf{x}_j|)
\end{align*}
Then $H$ is translationally invariant as well as rotationally invariant.
Hence, $[H, \mathbf{P}]=0$ and $[H,\mathbf{L}]=0$

\end{enumerate}
\end{problem}







% \end{multicols*}
\end{document}

