\documentclass[twoside,11pt]{article}
\usepackage[left=1in, right=1in, top=1in, bottom=1in]{geometry}
\usepackage{amsmath}
\usepackage{amssymb}
\usepackage{amsfonts}
\usepackage{mathtools}
\usepackage{amsthm}
\usepackage{fancyhdr}
\usepackage{enumitem}
\usepackage{siunitx}
\usepackage{booktabs}
\usepackage[hidelinks]{hyperref}
\usepackage{sectsty}
\usepackage{mathrsfs} % mathscr
\usepackage{tikz}
\usepackage{pgfplots}
\usepackage{multicol}
\usepackage{listings}
% \usepackage{amsart}
\usepackage{fontspec}
\usepackage{soul}


% allow H option of figure
\usepackage{float}

% math font (libertine)
\usepackage{libertinus-otf}

% braket
\usepackage{braket}

% physics
% \usepackage{physics}

% define latin modern font environment
\newcommand{\lms}{\fontfamily{lmss}\selectfont} % Latin Modern Roman
% \newcommand{\lmss}{\fontfamily{lmss}\selectfont} % Latin Modern Sans
% \newcommand{\lmss}{\fontfamily{lmtt}\selectfont} % Latin Modern Mono

% % change mathcal shape
% \usepackage[mathcal]{eucal}


% define math operators
\newcommand{\FF}{\mathbb{F}}
\newcommand{\RR}{\mathbb{R}}
\newcommand{\NN}{\mathbb{N}}
\newcommand{\ZZ}{\mathbb{Z}}
\newcommand{\QQ}{\mathbb{Q}}
\newcommand{\XX}{\mathbb{Y}}
\newcommand{\CL}{\mathcal{L}}
% \renewcommand{\d}{\mathrm{d}}
\renewcommand*\d{\mathop{}\!\mathrm{d}}
\DeclareMathOperator*{\argmax}{arg\,max}
\DeclareMathOperator*{\argmin}{arg\,min}
\DeclareMathOperator{\im}{im}
\DeclareMathOperator{\id}{id}
\DeclareMathOperator{\erf}{erf}
\renewcommand{\mod}[1]{\ (\mathrm{mod}\ #1)}

% section font style
\sectionfont{\sffamily\Large}
\subsectionfont{\sffamily\normalsize}
\subsubsectionfont{\bf}

% line spreading and break
\hyphenpenalty=5000
\tolerance=20
\setlength{\parindent}{0em}
\setlength\parskip{0.5em}
\allowdisplaybreaks
\linespread{0.9}

% enumerate settings
% no break before enumerate
\setlist[enumerate]{itemsep=2pt,topsep=2pt}

% theorem
% latex theorem
% definition style
\theoremstyle{definition}
\newtheorem{theorem}{Theorem}[subsection]
\newtheorem{axiom}{Axiom}[section]
\newtheorem{definition}{Definition}[section]
\newtheorem{example}{Example}[section]
\newtheorem{question}{Question}[section]
\newtheorem{exercise}{Exercise}[section]
\newtheorem*{exercise*}{Exercise}
\newtheorem{lemma}{Lemma}[section]
\newtheorem{proposition}{Proposition}[section]
\newtheorem{corollary}{Corollary}[section]
\newtheorem*{theorem*}{Theorem}
\newtheorem{problem}{Problem}
% remark style
\theoremstyle{remark}
\newtheorem*{remark}{Remark}
\newtheorem*{solution}{Solution}
\newtheorem*{claim}{Claim}


% paragraph indent
\setlength{\parindent}{0em}
\setlength\parskip{0.5em}

\newcommand\Code{PHY5410 FA22}
\newcommand\Ass{HW10}
\newcommand\name{Haoran Sun}
\newcommand\mail{haoransun@link.cuhk.edu.cn}

\title{{\lms \Code \ \Ass}}
\author{\lms \name \ (\href{mailto:\mail}{\mail})}
\date{\sffamily \today}

\makeatletter
% \let\Title\@title
\let\theauthor\@author
\let\thedate\@date

\fancypagestyle{plain}{%
    \fancyhf{}
    \lhead{\sffamily \Ass}
    \rhead{\sffamily \name}
    \rfoot{\sffamily\thepage}

    % # 页脚自定义
    \fancyfoot[L]{
        \begin{minipage}[c]{0.06\textwidth}
            \includegraphics[height=7.5mm]{logo2.png}
        \end{minipage}
    }
}
\fancypagestyle{title}{%
    \fancyhf{}
    \renewcommand{\headrulewidth}{0pt}
    % \lhead{\Title}
    % \rhead{\theauthor}
    \rfoot{\sffamily\thepage}

    % # 页脚自定义
    \fancyfoot[L]{
        \begin{minipage}[c]{0.06\textwidth}
            \includegraphics[height=7.5mm]{logo2.png}
        \end{minipage}
    }
}
\fancyfootoffset[L]{0.3cm}

% re-define title format
\makeatletter
\renewcommand{\maketitle}{\bgroup\setlength{\parindent}{0pt}
\begin{flushleft}
  \textbf{\Large\@title}

  \@author
\end{flushleft}\egroup
}
\makeatother

\pagestyle{plain}

% lstlisting settings
\lstset{
    basicstyle=\linespread{0.7}\footnotesize,
    breaklines=true,
    basewidth=0.5em
}


\begin{document}
\maketitle
\thispagestyle{title}
% \begin{multicols*}{2}

% \begin{remark}
%     $V_\epsilon(x)$ is used to denote a $\epsilon$-neighborhood
%     \begin{align*}
%         V_\epsilon(x) = B_\epsilon(x)\setminus\{x\}
%     \end{align*}
% \end{remark}


\begin{problem}[18.2]
Let the solution be in the form
\begin{align*}
    \psi_l(\mathbf{x}) &= R_l(r)Y_{lm}(\theta, \phi)
\end{align*}
In this case, we know that 
\begin{align*}
    R_l(r) &= \frac{u_l(r)}{r} = \begin{cases}
        A j_l(kr) & r < a\\
        B j_l(kr+\delta_l) & r \geq a
    \end{cases}
\end{align*}
where $k=\sqrt{2mE/\hbar^2}$.
\begin{enumerate}[label=(\alph*)]
\item By the continuity wrt $r$ and $p_r=-i\hbar(\partial_r + 1/r)$ of $R(r)$
at $r=a$, we have
\begin{align*}
    Aj_l(ka) &= B j_l(ka+\delta_l)\\
    \lim_{\epsilon\rightarrow 0}
    \int_{a-\epsilon}^{a+\epsilon} \d r\ H\psi(\mathbf{x}) &= 
    \lim_{\epsilon\rightarrow 0}
    \int_{a-\epsilon}^{a+\epsilon} \d r\ E\psi(\mathbf{x})\\
    \Rightarrow
    \lambda Aj_l(ka) &= 
    Akj_l'(ka) - Bkj_l'(ka+\delta_l)
\end{align*}
note that $j_l'$ does not denote the derivative but denotes the radial derivative
\begin{align*}
    j_l'(\varrho) &= \left[\frac{1}{\varrho}\frac{\d}{\d\varrho}\varrho\right] j_l(\varrho)
\end{align*}
Substituting $A/B = j_l(ka+\delta_l)/j_l(ka)$, we get an equation
\begin{align*}
    \lambda j_k(ka+\delta_l) &= 
    kj_l(ka+\delta_l)\frac{j'_l(ka)}{j_l(ka)} - 
    kj_l'(ka+\delta_l)\\
    \Rightarrow
    \frac{j_l'(ka+\delta_l)}{j_l(ka+\delta_l)} &= 
    \frac{j_l'(ka)}{j_k(ka)} - \frac{\lambda}{k}
\end{align*}
Assume the asymptotic solution applies here, where
\begin{align*}
    j_l(\varrho) &= \frac{1}{\varrho}\sin\left(\varrho - \frac{l\pi}{2}\right)\quad
    j_l'(\varrho) = \frac{1}{\varrho}\cos\left(\varrho - \frac{l\pi}{2}\right)
\end{align*}
Hence 
\begin{align*}
    \cot (ka+\delta_l-l\pi/2) &= 
    \cot (ka-l\pi/2) - \frac{\lambda}{k}\\
    \Rightarrow
    \delta_l &= \arctan\left[\frac{k}{k\cot (ka-l\pi/2) - \lambda}\right] 
    - ka + \frac{l\pi}{2}
\end{align*}
if we define $\xi=ka$, $g=\lambda a$, then
\begin{align*}
    \delta_l &= \arctan\left[\frac{\xi}{\xi\cot (\xi-l\pi/2) - g}\right] 
    - \xi + \frac{l\pi}{2}
\end{align*}

\item For $l=0$ we have
\begin{align*}
    \sigma_0 &= 
    \frac{4\pi}{k^2}\sin^2\delta_l
\end{align*}

\item The condition for $\delta_0$ to take its maximum is the same as
for $|\sin\delta_l/k|$ to take its maximum.

\item Given $g=\lambda a$ large enough, the $\arctan$ term in the
$\delta_l$ will vanish. Hence 
\begin{align*}
    \delta_l(k) &= -ka + \frac{l\pi}{2}
\end{align*}
Hence 
\begin{align*}
    \sigma_0(k) &= \frac{4\pi}{k^2}\sin^2(ka)
\end{align*}
Then easy to verify that
\begin{align*}
    \sigma_0(k) &\leq \lim_{k'\rightarrow 0}\sigma_0(k')
    \sigma_0(k') = 4\pi a^2
\end{align*}
then $\sigma_0$ approximates its maximum $4\pi a^2$ when $k$ is sufficiently small.

\end{enumerate}
\end{problem}



\begin{problem}[18.4]
The radial part of the wave function is
\begin{align*}
    R_l(a) &= \begin{cases}
        0 & r < a\\
        Aj_l(kr) + Bn_l(kr) & r \geq a
    \end{cases}
\end{align*}
To satisfy the continuity
\begin{align*}
    R_l(a) &= A j_l(ka) + B n_l(ka) = C j_l(ka+\delta_l) = 0\\
    \Rightarrow
    \delta_l &= \arctan \frac{j_l(ka)}{n_l(ka)} \approx
    ka - \frac{l\pi}{2}
\end{align*}
For $l=0$, we have $\delta_0=ka$.
Then 
\begin{align*}
    \sigma_0 &= \frac{4\pi}{k^2}\sin^2 ka
\end{align*}
$\sigma_0\approx 4\pi a^2$ when $ka$ small.
\end{problem}



\begin{problem}[18.5]
Let the potential be
\begin{align*}
    V(r) &= \begin{cases}
        V & r < a\\
        0 & r\geq a
    \end{cases}
\end{align*}
Hence the solution could be written as
\begin{align*}
    \psi(\mathbf{x}) &= R_l(r)Y_{lm}(\theta,\phi)\quad
    R_l = \begin{cases}
        Aj_l(qr) & r < a\\
        Bj_l(kr+\delta_l) & r \geq a
    \end{cases}
\end{align*}
where $q=\sqrt{2m(E-V)/\hbar^2}$, $k=\sqrt{2mE/\hbar^2}$.
According to the continuity of $R(r)$ wrt $r$ and $p_r$ at $r=a$, we have
\begin{align*}
    Aj_l(qa) &= Bj_l(ka+\delta_l)\quad 
    Aqj_l'(qa) = Bkj_l'(kr+\delta_l)
    \Rightarrow
    \frac{j_l'(ka+\delta_l)}{j_l(ka+\delta_l)} = 
    \frac{q}{k}\frac{j_l'(qa)}{j_l(qa)}
\end{align*}
Plugin $l=0$, we have
\begin{align*}
    \tan(ka+\delta_l) &= \frac{k}{q}\tan qa\\
    \Rightarrow \delta_l &= 
    \arctan\left[
        \frac{k}{q}\tan qa
    \right] - ka
\end{align*}

\end{problem}





% \end{multicols*}
\end{document}

