\documentclass[twoside,11pt]{article}
\usepackage[left=1in, right=1in, top=1in, bottom=1in]{geometry}
\usepackage{amsmath}
\usepackage{amssymb}
\usepackage{amsfonts}
\usepackage{mathtools}
\usepackage{amsthm}
\usepackage{fancyhdr}
\usepackage{enumitem}
\usepackage{siunitx}
\usepackage{booktabs}
\usepackage[hidelinks]{hyperref}
\usepackage{sectsty}
\usepackage{mathrsfs} % mathscr
\usepackage{tikz}
\usepackage{pgfplots}
\usepackage{multicol}
\usepackage{listings}
% \usepackage{amsart}
\usepackage{fontspec}


% allow H option of figure
\usepackage{float}

% math font (libertine)
\usepackage{libertinus-otf}

% braket
\usepackage{braket}

% define latin modern font environment
\newcommand{\lms}{\fontfamily{lmss}\selectfont} % Latin Modern Roman
% \newcommand{\lmss}{\fontfamily{lmss}\selectfont} % Latin Modern Sans
% \newcommand{\lmss}{\fontfamily{lmtt}\selectfont} % Latin Modern Mono

% % change mathcal shape
% \usepackage[mathcal]{eucal}


% define math operators
\newcommand{\F}{\mathbb{F}}
\newcommand{\R}{\mathbb{R}}
\newcommand{\N}{\mathbb{N}}
\newcommand{\Z}{\mathbb{Z}}
\newcommand{\Q}{\mathbb{Q}}
\newcommand{\X}{\mathbb{Y}}
\renewcommand{\L}{\mathcal{L}}
% \renewcommand{\d}{\mathrm{d}}
\renewcommand*\d{\mathop{}\!\mathrm{d}}
\DeclareMathOperator*{\argmax}{arg\,max}
\DeclareMathOperator*{\argmin}{arg\,min}
\DeclareMathOperator{\im}{im}
\DeclareMathOperator{\id}{id}
\renewcommand{\mod}[1]{\ (\mathrm{mod}\ #1)}

% section font style
\sectionfont{\sffamily\Large}
\subsectionfont{\sffamily\normalsize}
\subsubsectionfont{\bf}

% line spreading and break
\hyphenpenalty=5000
\tolerance=20
\setlength{\parindent}{0em}
\setlength\parskip{0.5em}
\allowdisplaybreaks
\linespread{0.9}

% theorem
% latex theorem
% definition style
\theoremstyle{definition}
\newtheorem{theorem}{Theorem}[subsection]
\newtheorem{axiom}{Axiom}[section]
\newtheorem{definition}{Definition}[section]
\newtheorem{example}{Example}[section]
\newtheorem{question}{Question}[section]
\newtheorem{exercise}{Exercise}[section]
\newtheorem*{exercise*}{Exercise}
\newtheorem{lemma}{Lemma}[section]
\newtheorem{proposition}{Proposition}[section]
\newtheorem{corollary}{Corollary}[section]
\newtheorem*{theorem*}{Theorem}
\newtheorem{problem}{Problem}
% remark style
\theoremstyle{remark}
\newtheorem*{remark}{Remark}
\newtheorem*{solution}{Solution}
\newtheorem*{claim}{Claim}


% paragraph indent
\setlength{\parindent}{0em}
\setlength\parskip{0.5em}

\newcommand\Code{PHY5410 FA22}
\newcommand\Ass{HW\#02}
\newcommand\name{Haoran Sun}
\newcommand\mail{haoransun@link.cuhk.edu.cn}

\title{{\lms \Code \ \Ass}}
\author{\lms \name \ (\href{mailto:\mail}{\mail})}
\date{\sffamily \today}

\makeatletter
% \let\Title\@title
\let\theauthor\@author
\let\thedate\@date

\fancypagestyle{plain}{%
    \fancyhf{}
    \lhead{\sffamily \Ass}
    \rhead{\sffamily \name}
    \rfoot{\sffamily\thepage}

    % # 页脚自定义
    \fancyfoot[L]{
        \begin{minipage}[c]{0.06\textwidth}
            \includegraphics[height=7.5mm]{logo2.png}
        \end{minipage}
    }
}
\fancypagestyle{title}{%
    \fancyhf{}
    \renewcommand{\headrulewidth}{0pt}
    % \lhead{\Title}
    % \rhead{\theauthor}
    \rfoot{\sffamily\thepage}

    % # 页脚自定义
    \fancyfoot[L]{
        \begin{minipage}[c]{0.06\textwidth}
            \includegraphics[height=7.5mm]{logo2.png}
        \end{minipage}
    }
}
\fancyfootoffset[L]{0.3cm}

% re-define title format
\makeatletter
\renewcommand{\maketitle}{\bgroup\setlength{\parindent}{0pt}
\begin{flushleft}
  \textbf{\Large\@title}

  \@author
\end{flushleft}\egroup
}
\makeatother

\pagestyle{plain}

% lstlisting settings
\lstset{
    basicstyle=\linespread{0.7}\footnotesize,
    breaklines=true,
    basewidth=0.5em
}


\begin{document}
\maketitle
\thispagestyle{title}
% \begin{multicols*}{2}

% \begin{remark}
%     $V_\epsilon(x)$ is used to denote a $\epsilon$-neighborhood
%     \begin{align*}
%         V_\epsilon(x) = B_\epsilon(x)\setminus\{x\}
%     \end{align*}
% \end{remark}

\begin{problem}[3.10]
\end{problem}


\begin{problem}[3.15]
Since
\begin{align*}
    \frac{(\Delta p)^2}{2m} = c(\Delta x)^4\\
    \Delta p\Delta x\geq \frac{\hbar}{2}
\end{align*}
Then
\begin{align*}
    \Delta x \geq \left(\frac{\hbar^2}{8mc}\right)^{1/6}
\end{align*}
\end{problem}


\begin{problem}[4.2]
    Using the fact that the energy uncertainty of a free wave packet is
    $\Delta E = p_0\Delta p/m$ and the time uncertainty is
    $\Delta t = m\Delta x/p_0$.
    Then 
    \begin{align*}
        \Delta E &= \frac{\hbar}{2dm}p_0\\
        \Delta E\Delta t &= \sqrt{1+\left(\frac{t\hbar}{2md^2}\right)^2}\frac{\hbar}{2}
    \end{align*}
\end{problem}


\begin{problem}[5.3]
\begin{claim}
    $L_i = L_i^\dagger$
\end{claim}
\begin{proof}
    Using the fact $[x_i,p_j]=i\hbar\delta_{ij}$
    \begin{align*}
        L_i^\dag &= [\epsilon_{ijk}(x_jp_k - x_kp_j)]^\dag\\
        &= \epsilon_{ijk}(p_k^\dag x_j^\dag - p_j^\dag x_k^\dag)\\
        &= \epsilon_{ijk}(p_k^\dag x_j^\dag - p_j^\dag x_k^\dag)\\
        &= \epsilon_{ijk}(x_jp_k - x_kp_j) = L_i\qedhere
    \end{align*}
\end{proof}
\begin{claim}
    $\braket{\psi|L_i^2\psi}\geq 0$.
\end{claim}
\begin{proof}
    Since $L_i=L_i^\dagger$, we have
    \begin{align*}
        \braket{\psi|L_i^2\psi} &= \braket{L_i^\dagger\psi|L_i\psi}\\
        &= \braket{L_i\psi|L_i\psi} \geq 0\qedhere
    \end{align*}
\end{proof}

By the claims above, we can derive that
\begin{align*}
    \braket{\psi|\mathbf{L}^2\psi} = 0\Leftrightarrow
    \sum_{i}\braket{\psi|L_i^2\psi} = 0\Leftrightarrow
    \braket{\psi|L_i^2\psi} = \braket{L_i\psi|L_i\psi} = 0
    \Leftrightarrow \ket{L_i\psi} = \ket{0}\Rightarrow
    \braket{\psi|L_i\psi} = 0
\end{align*}
note that we can derive $\ket{L_i\psi}=0$ 
if $\ket{L_i\psi}$ is indeed a continuous function under any
representations.


\end{problem}


% \end{multicols*}
\end{document}

