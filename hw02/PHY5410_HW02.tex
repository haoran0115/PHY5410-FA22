\documentclass[twoside,11pt]{article}
\usepackage[left=1in, right=1in, top=1in, bottom=1in]{geometry}
\usepackage{amsmath}
\usepackage{amssymb}
\usepackage{amsfonts}
\usepackage{mathtools}
\usepackage{amsthm}
\usepackage{fancyhdr}
\usepackage{enumitem}
\usepackage{siunitx}
\usepackage{booktabs}
\usepackage[hidelinks]{hyperref}
\usepackage{sectsty}
\usepackage{mathrsfs} % mathscr
\usepackage{tikz}
\usepackage{pgfplots}
\usepackage{multicol}
\usepackage{listings}
% \usepackage{amsart}
\usepackage{fontspec}


% allow H option of figure
\usepackage{float}

% math font (libertine)
\usepackage{libertinus-otf}

% braket
\usepackage{braket}

% define latin modern font environment
\newcommand{\lms}{\fontfamily{lmss}\selectfont} % Latin Modern Roman
% \newcommand{\lmss}{\fontfamily{lmss}\selectfont} % Latin Modern Sans
% \newcommand{\lmss}{\fontfamily{lmtt}\selectfont} % Latin Modern Mono

% % change mathcal shape
% \usepackage[mathcal]{eucal}


% define math operators
\newcommand{\F}{\mathbb{F}}
\newcommand{\R}{\mathbb{R}}
\newcommand{\N}{\mathbb{N}}
\newcommand{\Z}{\mathbb{Z}}
\newcommand{\Q}{\mathbb{Q}}
\newcommand{\X}{\mathbb{Y}}
\renewcommand{\L}{\mathcal{L}}
% \renewcommand{\d}{\mathrm{d}}
\renewcommand*\d{\mathop{}\!\mathrm{d}}
\DeclareMathOperator*{\argmax}{arg\,max}
\DeclareMathOperator*{\argmin}{arg\,min}
\DeclareMathOperator{\im}{im}
\DeclareMathOperator{\id}{id}
\renewcommand{\mod}[1]{\ (\mathrm{mod}\ #1)}

% section font style
\sectionfont{\sffamily\Large}
\subsectionfont{\sffamily\normalsize}
\subsubsectionfont{\bf}

% line spreading and break
\hyphenpenalty=5000
\tolerance=20
\setlength{\parindent}{0em}
\setlength\parskip{0.5em}
\allowdisplaybreaks
\linespread{0.9}

% theorem
% latex theorem
% definition style
\theoremstyle{definition}
\newtheorem{theorem}{Theorem}[subsection]
\newtheorem{axiom}{Axiom}[section]
\newtheorem{definition}{Definition}[section]
\newtheorem{example}{Example}[section]
\newtheorem{question}{Question}[section]
\newtheorem{exercise}{Exercise}[section]
\newtheorem*{exercise*}{Exercise}
\newtheorem{lemma}{Lemma}[section]
\newtheorem{proposition}{Proposition}[section]
\newtheorem{corollary}{Corollary}[section]
\newtheorem*{theorem*}{Theorem}
\newtheorem{problem}{Problem}
% remark style
\theoremstyle{remark}
\newtheorem*{remark}{Remark}
\newtheorem*{solution}{Solution}
\newtheorem*{claim}{Claim}


% paragraph indent
\setlength{\parindent}{0em}
\setlength\parskip{0.5em}

\newcommand\Code{PHY5410 FA22}
\newcommand\Ass{HW02}
\newcommand\name{Haoran Sun}
\newcommand\mail{haoransun@link.cuhk.edu.cn}

\title{{\lms \Code \ \Ass}}
\author{\lms \name \ (\href{mailto:\mail}{\mail})}
\date{\sffamily \today}

\makeatletter
% \let\Title\@title
\let\theauthor\@author
\let\thedate\@date

\fancypagestyle{plain}{%
    \fancyhf{}
    \lhead{\sffamily \Ass}
    \rhead{\sffamily \name}
    \rfoot{\sffamily\thepage}

    % # 页脚自定义
    \fancyfoot[L]{
        \begin{minipage}[c]{0.06\textwidth}
            \includegraphics[height=7.5mm]{logo2.png}
        \end{minipage}
    }
}
\fancypagestyle{title}{%
    \fancyhf{}
    \renewcommand{\headrulewidth}{0pt}
    % \lhead{\Title}
    % \rhead{\theauthor}
    \rfoot{\sffamily\thepage}

    % # 页脚自定义
    \fancyfoot[L]{
        \begin{minipage}[c]{0.06\textwidth}
            \includegraphics[height=7.5mm]{logo2.png}
        \end{minipage}
    }
}
\fancyfootoffset[L]{0.3cm}

% re-define title format
\makeatletter
\renewcommand{\maketitle}{\bgroup\setlength{\parindent}{0pt}
\begin{flushleft}
  \textbf{\Large\@title}

  \@author
\end{flushleft}\egroup
}
\makeatother

\pagestyle{plain}

% lstlisting settings
\lstset{
    basicstyle=\linespread{0.7}\footnotesize,
    breaklines=true,
    basewidth=0.5em
}


\begin{document}
\maketitle
\thispagestyle{title}
% \begin{multicols*}{2}

% \begin{remark}
%     $V_\epsilon(x)$ is used to denote a $\epsilon$-neighborhood
%     \begin{align*}
%         V_\epsilon(x) = B_\epsilon(x)\setminus\{x\}
%     \end{align*}
% \end{remark}

\begin{problem}[3.10]
    The wavefunction under the momentum representation would be 
    \begin{align*}
        \varphi(p, t) &= \frac{1}{\sqrt{2\pi\hbar}}\int\d x\ \psi(x, t)e^{-ipx/\hbar}\\
        &= \frac{1}{\sqrt{2\pi\hbar}}\frac{1}{2\pi\hbar}
        \iint\d x\d p'\ g(p') e^{ix(p-p')/\hbar}e^{-iE(p')t/\hbar}e^{i\alpha(p')}\\
        &= \frac{1}{\sqrt{2\pi\hbar}}\int\d p'\ g(p')e^{-iE(p')t/\hbar}e^{i\alpha(p')}\delta(p-p')\\
        &= \frac{1}{\sqrt{2\pi\hbar}} g(p)e^{-iE(p)t/\hbar}e^{i\alpha(p)}
    \end{align*}
    and
    \begin{align*}
        |\varphi(p, t)|^2 &= \frac{1}{2\pi\hbar}g^2(p)
    \end{align*}
    Therefore
    \begin{align*}
        \braket{p} &= \braket{\varphi|p \varphi} = \frac{1}{2\pi\hbar}\int\d p\ pg^2(p) = p_0\\
        \braket{p^2} &= \braket{\varphi|p^2\varphi} = \frac{1}{2\pi\hbar}\int\d p\ p^2g^2(p)
    \end{align*}
    Using the fact that $x = ih\d/\d p$ under momentum representation, we have
    \begin{align*}
        x\varphi(p, t) &= \frac{1}{\sqrt{2\pi\hbar}} g(p)(E'(p)t - \alpha'(p)\hbar)
        e^{-iE(t)/\hbar + i\alpha(p)} + \frac{1}{\sqrt{2\pi\hbar}}i\hbar g'(p)e^{-iE(t)/\hbar + i\alpha(p)}
    \end{align*}
    Thus
    \begin{align*}
        \braket{x} &= \braket{\varphi|x\varphi} \\
        &= \frac{1}{2\pi\hbar}\int\d p\
        [E'(p)-\alpha'(p)\hbar] g^2(p) + i\hbar g(p)g'(p)\\
        &= \frac{1}{2\pi\hbar}\int\d p\ [E'(p)-\alpha'(p)\hbar] g^2(p)\\
        &= \braket{E'(p)t-\alpha'(p)\hbar}
    \end{align*}
    note that the term $g(p)g'(p)$ vanishes under integration due to the boundary condition.
    \begin{align*}
        \int\d p\ g(p)g'(p) = \left. \frac{1}{2}g^2(p)\right |_\Omega = 0
    \end{align*}
    For $x^2$ (ignoring $g(p)g'(p)$ term)
    \begin{align*}
        \braket{x^2} &= \braket{x\varphi|x\varphi}\\
        &= \frac{1}{2\pi\hbar}\int\d p\ \hbar^2 {g'}^2(p) + [E'(p)t-\alpha(p)\hbar]^2 g^2(p)\\
        &= \braket{[E'(p)t-\alpha'(p)\hbar]^2} + \frac{1}{2\pi\hbar}\int\d p\ 
        \hbar^2 {g'}^2(p)
    \end{align*}
    Plugin a gaussian wave packet with $\sigma_p=\Delta p$, i.e.
    \begin{align*}
        g(p) &= Ae^{\frac{-(p-p_0)^2}{4\Delta p^2}}\\
        E(p) &= \frac{p^2}{2m}\\
        \alpha(p) &= 0
    \end{align*}
    we have
    \begin{align*}
        \braket{p} &= p_0\\
        \braket{p^2} &= \Delta p^2 + p_0^2\\
        \braket{x} &= \braket{pt/m} = \frac{t}{m}\braket{p} = \frac{p_0t}{m}\\
        \braket{x^2} &= \braket{p^2t^2/m^2} +
        \braket{\hbar^2\left(\frac{p-p_0}{2\Delta p^2}\right)^2}\\
        &= \frac{t^2}{m^2}\braket{p^2} + \frac{\hbar^2}{4\Delta p^4}\braket{(p-p_0)^2}\\
        &= \frac{t^2}{m^2}(p_0^2 + \Delta p^2) + \frac{\hbar^2}{4\Delta p^2}
    \end{align*}
    In this manner, we have
    \begin{align*}
        \Delta x &= \sqrt{\frac{t^2}{m^2}\Delta p^2 + \frac{\hbar^2}{4\Delta p^2}}
        = \frac{\hbar}{2\Delta p}\sqrt{1 + \frac{4t^2\Delta p^4}{m^2\hbar^2}}\\
        \Rightarrow
        \Delta x\Delta p &= 
        \frac{\hbar}{2}\sqrt{1 + \frac{4t^2\Delta p^4}{m^2\hbar^2}}\geq \frac{\hbar}{2}
    \end{align*}
\end{problem}


\begin{problem}[3.15]
The energy of the system is
\begin{align*}
    E = T + V = \frac{p^2}{2m} + cx^4
\end{align*}
Substitute $p$ by $px = \hbar/2$
\begin{align*}
    E = \frac{\hbar^2}{8mx^2} + cx^4
\end{align*}
Find the minimum of $E$ by letting $\d E/\d x=0$
\begin{align*}
    \frac{\d E}{\d x} = 4cx^3 - \frac{\hbar^2}{4mx^3} = 0\Rightarrow
    x_0=\left(\frac{\hbar^2}{16mc}\right)^{1/6}
\end{align*}
Then the ground state energy approximately equals to 
\begin{align*}
    E_0 = \frac{\hbar^2}{8mx_0^2} + cx_0^4 = \frac{\hbar^2}{8m}\left(
        \frac{16mc}{\hbar^2}
    \right)^{1/3}
    + c\left(\frac{\hbar^2}{16mc}\right)^{2/3} 
    = 3c\left(\frac{\hbar^2}{16mc}\right)^{2/3}
\end{align*}
\end{problem}


\begin{problem}[4.2]
    Using the fact that the energy uncertainty of a free wave packet is
    $\Delta E = p_0\Delta p/m$ and the time uncertainty is
    $\Delta t = m\Delta x/p_0$.
    Then 
    \begin{align*}
        \Delta E &= \frac{\hbar}{2dm}p_0\\
        \Delta E\Delta t &= \sqrt{1+\left(\frac{t\hbar}{2md^2}\right)^2}\frac{\hbar}{2}
    \end{align*}
\end{problem}


\begin{problem}[5.3]
\begin{claim}
    $L_i = L_i^\dagger$
\end{claim}
\begin{proof}
    Using the fact $[x_i,p_j]=i\hbar\delta_{ij}$ ($x_i$, $p_j$ commute when $i\neq j$)
    \begin{align*}
        L_i^\dag &= \epsilon_{ijk}p_k^\dag x_j^\dag = 
        \epsilon_{ijk}p_kx_j = \epsilon_{ijk} x_jp_k = L_i\qedhere
    \end{align*}
\end{proof}
\begin{claim}
    $\braket{\psi|L_i^2\psi}\geq 0$.
\end{claim}
\begin{proof}
    Since $L_i=L_i^\dagger$, we have
    \begin{align*}
        \braket{\psi|L_i^2\psi} &= \braket{L_i^\dagger\psi|L_i\psi} 
        = \braket{L_i\psi|L_i\psi} \geq 0\qedhere
    \end{align*}
\end{proof}

By the claims above, we can derive that
\begin{align*}
    \braket{\psi|\mathbf{L}^2\psi} = 0\Leftrightarrow
    \sum_{i}\braket{\psi|L_i^2\psi} = 0\Leftrightarrow
    \braket{\psi|L_i^2\psi} = \braket{L_i\psi|L_i\psi} = 0
    \Leftrightarrow \ket{L_i\psi} = \ket{0}\Rightarrow
    \braket{\psi|L_i\psi} = 0
\end{align*}
note that we can derive $\ket{L_i\psi}=\ket{0}$ 
if $\ket{L_i\psi}$ is indeed a continuous function under any
representations.


\end{problem}


% \end{multicols*}
\end{document}

