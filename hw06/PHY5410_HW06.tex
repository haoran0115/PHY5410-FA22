\documentclass[twoside,11pt]{article}
\usepackage[left=1in, right=1in, top=1in, bottom=1in]{geometry}
\usepackage{amsmath}
\usepackage{amssymb}
\usepackage{amsfonts}
\usepackage{mathtools}
\usepackage{amsthm}
\usepackage{fancyhdr}
\usepackage{enumitem}
\usepackage{siunitx}
\usepackage{booktabs}
\usepackage[hidelinks]{hyperref}
\usepackage{sectsty}
\usepackage{mathrsfs} % mathscr
\usepackage{tikz}
\usepackage{pgfplots}
\usepackage{multicol}
\usepackage{listings}
% \usepackage{amsart}
\usepackage{fontspec}
\usepackage{soul}


% allow H option of figure
\usepackage{float}

% math font (libertine)
\usepackage{libertinus-otf}

% braket
\usepackage{braket}

% physics
% \usepackage{physics}

% define latin modern font environment
\newcommand{\lms}{\fontfamily{lmss}\selectfont} % Latin Modern Roman
% \newcommand{\lmss}{\fontfamily{lmss}\selectfont} % Latin Modern Sans
% \newcommand{\lmss}{\fontfamily{lmtt}\selectfont} % Latin Modern Mono

% % change mathcal shape
% \usepackage[mathcal]{eucal}


% define math operators
\newcommand{\FF}{\mathbb{F}}
\newcommand{\RR}{\mathbb{R}}
\newcommand{\NN}{\mathbb{N}}
\newcommand{\ZZ}{\mathbb{Z}}
\newcommand{\QQ}{\mathbb{Q}}
\newcommand{\XX}{\mathbb{Y}}
\newcommand{\CL}{\mathcal{L}}
% \renewcommand{\d}{\mathrm{d}}
\renewcommand*\d{\mathop{}\!\mathrm{d}}
\DeclareMathOperator*{\argmax}{arg\,max}
\DeclareMathOperator*{\argmin}{arg\,min}
\DeclareMathOperator{\im}{im}
\DeclareMathOperator{\id}{id}
\renewcommand{\mod}[1]{\ (\mathrm{mod}\ #1)}

% section font style
\sectionfont{\sffamily\Large}
\subsectionfont{\sffamily\normalsize}
\subsubsectionfont{\bf}

% line spreading and break
\hyphenpenalty=5000
\tolerance=20
\setlength{\parindent}{0em}
\setlength\parskip{0.5em}
\allowdisplaybreaks
\linespread{0.9}

% theorem
% latex theorem
% definition style
\theoremstyle{definition}
\newtheorem{theorem}{Theorem}[subsection]
\newtheorem{axiom}{Axiom}[section]
\newtheorem{definition}{Definition}[section]
\newtheorem{example}{Example}[section]
\newtheorem{question}{Question}[section]
\newtheorem{exercise}{Exercise}[section]
\newtheorem*{exercise*}{Exercise}
\newtheorem{lemma}{Lemma}[section]
\newtheorem{proposition}{Proposition}[section]
\newtheorem{corollary}{Corollary}[section]
\newtheorem*{theorem*}{Theorem}
\newtheorem{problem}{Problem}
% remark style
\theoremstyle{remark}
\newtheorem*{remark}{Remark}
\newtheorem*{solution}{Solution}
\newtheorem*{claim}{Claim}


% paragraph indent
\setlength{\parindent}{0em}
\setlength\parskip{0.5em}

\newcommand\Code{PHY5410 FA22}
\newcommand\Ass{HW06}
\newcommand\name{Haoran Sun}
\newcommand\mail{haoransun@link.cuhk.edu.cn}

\title{{\lms \Code \ \Ass}}
\author{\lms \name \ (\href{mailto:\mail}{\mail})}
\date{\sffamily \today}

\makeatletter
% \let\Title\@title
\let\theauthor\@author
\let\thedate\@date

\fancypagestyle{plain}{%
    \fancyhf{}
    \lhead{\sffamily \Ass}
    \rhead{\sffamily \name}
    \rfoot{\sffamily\thepage}

    % # 页脚自定义
    \fancyfoot[L]{
        \begin{minipage}[c]{0.06\textwidth}
            \includegraphics[height=7.5mm]{logo2.png}
        \end{minipage}
    }
}
\fancypagestyle{title}{%
    \fancyhf{}
    \renewcommand{\headrulewidth}{0pt}
    % \lhead{\Title}
    % \rhead{\theauthor}
    \rfoot{\sffamily\thepage}

    % # 页脚自定义
    \fancyfoot[L]{
        \begin{minipage}[c]{0.06\textwidth}
            \includegraphics[height=7.5mm]{logo2.png}
        \end{minipage}
    }
}
\fancyfootoffset[L]{0.3cm}

% re-define title format
\makeatletter
\renewcommand{\maketitle}{\bgroup\setlength{\parindent}{0pt}
\begin{flushleft}
  \textbf{\Large\@title}

  \@author
\end{flushleft}\egroup
}
\makeatother

\pagestyle{plain}

% lstlisting settings
\lstset{
    basicstyle=\linespread{0.7}\footnotesize,
    breaklines=true,
    basewidth=0.5em
}


\begin{document}
\maketitle
\thispagestyle{title}
% \begin{multicols*}{2}

% \begin{remark}
%     $V_\epsilon(x)$ is used to denote a $\epsilon$-neighborhood
%     \begin{align*}
%         V_\epsilon(x) = B_\epsilon(x)\setminus\{x\}
%     \end{align*}
% \end{remark}

\begin{problem}[9.5]\
\begin{enumerate}[label=(\alph*)]
\item Since
\begin{align*}
    \frac{\d}{\d t}S_x(t) &= \frac{i}{\hbar}[H, S_x]_H
    = \frac{i}{\hbar}\frac{eB}{mc}[S_z, S_x]_H
    = -\frac{eB}{mc}S_y(t)\\
    \frac{\d}{\d t}S_y(t) &= \frac{i}{\hbar}[H, S_y]_H
    = \frac{i}{\hbar}\frac{eB}{mc}[S_z, S_y]_H
    = \frac{eB}{mc}S_x(t)\\
    \frac{\d}{\d t}S_z(t) &= \frac{i}{\hbar}[H, S_z]_H
    = \frac{i}{\hbar}\frac{eB}{mc}[S_z, S_z]_H
    = 0
\end{align*}
Thus we have a set of differential equations
\begin{align*}
    \frac{\d}{\d t}
    \begin{bmatrix}
        S_x(t) \\ S_y(t)
    \end{bmatrix}
    = \begin{bmatrix}
        0 & -eB/mc \\ eB/mc & 0
    \end{bmatrix}
    \begin{bmatrix}
        S_x(t) \\ S_y(t)
    \end{bmatrix}
\end{align*}
The solution is
\begin{align*}
    \begin{bmatrix}
        S_x(t)\\ S_y(t)
    \end{bmatrix} &= 
    \exp\left(
    \begin{bmatrix}
        0 & -\omega t \\ \omega t& 0
    \end{bmatrix}\right)
    \begin{bmatrix}
        S_x(0) \\ S_y(0)
    \end{bmatrix}
    = \begin{bmatrix}
        \cos\omega t\ S_x(0) - \sin\omega t\ S_y(0)\\
        \sin\omega t\ S_x(0) + \cos\omega t\ S_y(0)
    \end{bmatrix}
\end{align*}
where $\omega = eB/mc$.

\item Note that $S_z$ could be written in $\RR^{2\times 2}$
under the basis representation
\begin{align*}
    S_z &= \frac{\hbar}{2}\begin{bmatrix}
        1 & 0 \\ 0 & -1
    \end{bmatrix}
\end{align*}
Then
\begin{align*}
    \Psi(t) &= \exp(-i\omega S_zt/\hbar)\Psi(0)
    = \exp\left(
        -i\frac{\omega}{\hbar}\frac{\hbar}{2}\begin{bmatrix}
            1 & 0 \\ 0 & -1
        \end{bmatrix}t
    \right)\Psi(0)
    = \begin{bmatrix}
        e^{-i\omega t/2} & 0\\ 0 & e^{i\omega t/2}
    \end{bmatrix}\begin{bmatrix}
        a\\b
    \end{bmatrix}
    =
    \begin{bmatrix}
        ae^{-i\omega t/2}\\
        be^{i\omega t/2}
    \end{bmatrix}
\end{align*}

\item It should be clear that the probability of getting
$\ket{\uparrow}$ would be $|a|^2$ and $\ket{\downarrow}$ should
be $|b|^2$.
If the spin is oriented in the $x$ direction when $t=0$, 
then $[a\quad b]^T$ should be an eigenvector of $S_x$.
Let $a=b$ and $|a|=|b|=1/\sqrt[]{2}$, then we have the probability
of getting $\ket{\uparrow}$ equals to 
\begin{align*}
    P(\ket{\uparrow}) &= |\braket{\uparrow|\psi}|^2 = |ae^{-i\omega t/2}|^2 = |a|^2 = \frac{1}{2}
\end{align*}

\item Pick eigenfunction of $S_x$ where $S_x\ket{\uparrow_x} = \hbar/2\ket{\uparrow_x}$.
Let
\begin{align*}
    \ket{\uparrow_x} &= \frac{1}{\sqrt[]{2}}\ket{\uparrow} + \frac{1}{\sqrt[]{2}}\ket{\downarrow}
\end{align*}
Then the probability of getting $\ket{\uparrow_x}$ is
\begin{align*}
    P(\ket{\uparrow_x}) &= |\braket{\uparrow_x|\psi}^2|\\
    &= 
    \left[\frac{1}{\sqrt[]{2}}(ae^{-i\omega t/2} + be^{i\omega t/2})\right]^2\\
    &= \frac{1}{2}(|a|^2 + |b|^2 + a^*be^{i\omega t}+ab^*e^{-i\omega t})\\
    &= \frac{1}{2} + \frac{1}{2}\cos\omega t = \cos^2(\omega t/2)
\end{align*}
using the fact that $a=b$.

\item Using the equation 
\begin{align*}
    i\hbar\frac{\partial}{\partial t}\Psi &= \frac{e_0}{mc}\frac{\hbar}{2}BS_z\Psi
    \Rightarrow
    i\hbar\frac{\d}{\d t}\begin{bmatrix}
        a(t)\\ b(t)
    \end{bmatrix} 
    = \frac{e_0}{mc}\frac{\hbar}{2}B\sigma_z\begin{bmatrix}
        a(t)\\ b(t)
    \end{bmatrix}
\end{align*}
then it can be solved refer to (b).

% \item In a time-dependent, external electromagnetic field,
% the Pauli equation writes
% \begin{align*}
%     i\hbar\frac{\partial}{\partial t}
%     \begin{bmatrix}
%         \psi_+(x, t)\\ \psi_-(x,t)
%     \end{bmatrix}
%     &=\left[
%         \left(
%             \frac{1}{2m}\left(
%                 \frac{\hbar}{i}\nabla - \frac{e}{c}\mathbf{A}(x,t)
%             \right)^2
%             + e\Phi(x,t)
%         \right)\mathbb{1} + \mu_B\boldsymbol{\sigma}\cdot\mathbf{B}
%     \right] 
%     \begin{bmatrix}
%         \psi_+(x, t)\\ \psi_-(x,t)
%     \end{bmatrix}
% \end{align*}
% In this case, since $\mathbf{A}$ and $\mathbf{B}$ are independent from time,
% then it should be reasonable


\end{enumerate}
\end{problem}

\begin{problem}[10.2]
Note that
\begin{align*}
    \mathbf{S}_1\mathbf{S}_2 &= S_{1z} S_{2z} + \frac{1}{2}S_{1+}S_{2-}
    + \frac{1}{2}S_{1-}S_{2+}
\end{align*}
Let $\ket{\uparrow\uparrow}$, $\ket{\uparrow\downarrow}$,
$\ket{\downarrow\uparrow}$ and $\ket{\downarrow\downarrow}$ 
be a set of orthonormal basis.
Then we can derive that
\begin{align*}
    H\ket{\uparrow\uparrow} &= \left(
        -\frac{a+b}{2}B\hbar + \frac{1}{4}J\hbar^2
    \right)\ket{\uparrow\uparrow}\\
    H\ket{\downarrow\downarrow} &= 
    \left(
        \frac{a+b}{2}B\hbar+\frac{1}{4}J\hbar^2
    \right)\ket{\downarrow\downarrow}\\
    H\ket{\uparrow\downarrow} &=
    \left(
        -\frac{a-b}{2}B\hbar-\frac{1}{4}J\hbar^2
    \right)\ket{\uparrow\downarrow}
    + \frac{1}{2}J\hbar^2\ket{\downarrow\uparrow}\\
    H\ket{\downarrow\uparrow} &=
    \left(
        \frac{a-b}{2}B\hbar-\frac{1}{4}J\hbar^2
    \right)\ket{\downarrow\uparrow}
    + \frac{1}{2}J\hbar^2\ket{\uparrow\downarrow}
\end{align*}
To diagonalize the last two terms, we can consider the eigenvalues and
eigenvectors of the following matrix
\begin{align*}
    \begin{bmatrix}
        -\left(\dfrac{a-b}{2}B\hbar+\dfrac{1}{4}J\hbar^2\right) & 
        \dfrac{1}{2}J\hbar^2\\[1em]
        \dfrac{1}{2}J\hbar^2 & 
        \left(\dfrac{a-b}{2}B\hbar-\dfrac{1}{4}J\hbar^2\right)
    \end{bmatrix}
    \Rightarrow
    \lambda=-\frac{1}{4}J\hbar^2\pm\frac{1}{2}\sqrt[]{(a-b)^2B^2\hbar^2+J^2\hbar^4}
\end{align*}
Therefore we have four eigenvalues corresponding to four eigenstates.
\begin{align*}
    \lambda_1 &= -\frac{a+b}{2}B\hbar + \frac{1}{4}J\hbar^2\\
    \lambda_2 &= -\frac{1}{4}J\hbar^2 + \frac{1}{2}\sqrt[]{(a-b)^2B^2\hbar^2+J^2\hbar^4}\\
    \lambda_3 &= -\frac{1}{4}J\hbar^2 - \frac{1}{2}\sqrt[]{(a-b)^2B^2\hbar^2+J^2\hbar^4}\\
    \lambda_4 &= \frac{a+b}{2}B\hbar + \frac{1}{4}J\hbar^2
\end{align*}

\end{problem}


\begin{problem}[11.4]
Define $\ket{n_0}$ as the $n$th eigenvector of $H_0$, $\psi_0=\ket{0_0}$,
$H_1=x$, $\lambda=-eE$.
Using the fact that
\begin{align*}
    a_+ & = \frac{1}{\sqrt[]{2m\omega\hbar}}(m\omega x - ip)\\
    a_- & = \frac{1}{\sqrt[]{2m\omega\hbar}}(m\omega x + ip)\\
    [a_-, x] &= \frac{1}{\sqrt[]{2m\omega\hbar}}i[p,x] = \sqrt[]{\frac{\hbar}{2m\omega}}\\
    [a_-^n, x] &= na_-^{n-1}[a_-, x] = n\sqrt[]{\frac{\hbar}{2m\omega}}a_-^{n-1}
\end{align*}
The we can conclude that
\begin{align*}
    \braket{n_0|x|n_0} &= 0 \\
    \braket{q_0|x|n_0} &= \frac{1}{\sqrt[]{q! n!}}\braket{a_+^q\psi_0|x|a_+^n\psi_0} 
    = \frac{1}{\sqrt[]{q! n!}}\braket{\psi_0|a_-^qxa_+^n\psi_0}
    = \frac{1}{\sqrt[]{q! n!}}\braket{\psi_0|\left(xa_-^q+ q\sqrt[]{\frac{\hbar}{2m\omega}}a_-^{q-1}\right)a_+^n\psi_0}\\
    &= \frac{1}{\sqrt[]{q! n!}}q\sqrt[]{\frac{\hbar}{2m\omega}}\braket{\psi_0|a_-^{q-1} a_+^n\psi_0}
    = \frac{1}{\sqrt[]{q! n!}}q\sqrt[]{\frac{\hbar}{2m\omega}}\braket{a_+^{q-1}\psi_0| a_+^n\psi_0}
\end{align*}
Let $q>n$, $\braket{q_0|x|n_0}$ would vanish if $q\neq n+1$ due to the
orthogonality of eigenfunctions. Hence
\begin{align*}
    \braket{q_0|x|n_0} &=\begin{cases}
        \sqrt[]{\frac{(n+1)\hbar}{2m\omega}} & q = n+1\\
        0 & \text{otherwise}
    \end{cases}
\end{align*}
We can further generalize the conclusion to
\begin{align*}
    \braket{q_0|x|n_0} &=\begin{cases}
        \sqrt[]{\frac{(n+1)\hbar}{2m\omega}} & q = n+1\\
        \sqrt[]{\frac{n\hbar}{2m\omega}} & q = n-1\\
        0 & \text{otherwise}
    \end{cases}
\end{align*}

Then, we have
\begin{align*}
    E_n^0 &= \left(n+\frac{1}{2}\right)\hbar\omega\\
    E^1_n &= \braket{n_0|H_1|n_0} = 0\\
    E^2_n &= -\frac{(n+1)\hbar}{2m\omega}\frac{1}{\hbar\omega} + \frac{n\hbar}{2m\omega}\frac{1}{\hbar\omega}
    = -\frac{1}{2m\omega^2}\\
    \ket{n_1} &= -\frac{1}{\hbar\omega}\sqrt[]{\frac{(n+1)\hbar}{2m\omega}}\ket{(n+1)_0}
    + \frac{1}{\hbar\omega}\sqrt[]{\frac{n\hbar}{2m\omega}}\ket{(n-1)_0}
\end{align*}
The energy be expanded to second-order equals to 
\begin{align*}
    E &= E_0 + \lambda E_1 + \lambda^2 E_2 
    = \left(n+\frac{1}{2}\right)\hbar\omega - \frac{e^2E^2}{2m\omega^2}
\end{align*}
By some simple algebras, we can show the exact result is
\begin{align*}
    H &= -\frac{\hbar^2}{2m}\frac{\d^2}{\d x^2}
    + \frac{m\omega^2}{2}\left(x-\frac{eE}{m\omega^2}\right)^2
    - \frac{e^2E^2}{2m\omega^2}\Rightarrow
    E_n = \left(n+\frac{1}{2}\right)\hbar\omega - \frac{e^2E^2}{2m\omega^2}
\end{align*}
Then we can see that it is the second-order approximation is the same as the exact result.
\end{problem}





% \end{multicols*}
\end{document}

