\documentclass[twoside,11pt]{article}
\usepackage[left=1in, right=1in, top=1in, bottom=1in]{geometry}
\usepackage{amsmath}
\usepackage{amssymb}
\usepackage{amsfonts}
\usepackage{mathtools}
\usepackage{amsthm}
\usepackage{fancyhdr}
\usepackage{enumitem}
\usepackage{siunitx}
\usepackage{booktabs}
\usepackage[hidelinks]{hyperref}
\usepackage{sectsty}
\usepackage{mathrsfs} % mathscr
\usepackage{tikz}
\usepackage{pgfplots}
\usepackage{multicol}
\usepackage{listings}
% \usepackage{amsart}
\usepackage{fontspec}
\usepackage{soul}


% allow H option of figure
\usepackage{float}

% math font (libertine)
\usepackage{libertinus-otf}

% braket
\usepackage{braket}

% physics
% \usepackage{physics}

% define latin modern font environment
\newcommand{\lms}{\fontfamily{lmss}\selectfont} % Latin Modern Roman
% \newcommand{\lmss}{\fontfamily{lmss}\selectfont} % Latin Modern Sans
% \newcommand{\lmss}{\fontfamily{lmtt}\selectfont} % Latin Modern Mono

% % change mathcal shape
% \usepackage[mathcal]{eucal}


% define math operators
\newcommand{\FF}{\mathbb{F}}
\newcommand{\RR}{\mathbb{R}}
\newcommand{\NN}{\mathbb{N}}
\newcommand{\ZZ}{\mathbb{Z}}
\newcommand{\QQ}{\mathbb{Q}}
\newcommand{\XX}{\mathbb{Y}}
\newcommand{\CL}{\mathcal{L}}
% \renewcommand{\d}{\mathrm{d}}
\renewcommand*\d{\mathop{}\!\mathrm{d}}
\DeclareMathOperator*{\argmax}{arg\,max}
\DeclareMathOperator*{\argmin}{arg\,min}
\DeclareMathOperator{\im}{im}
\DeclareMathOperator{\id}{id}
\DeclareMathOperator{\erf}{erf}
\renewcommand{\mod}[1]{\ (\mathrm{mod}\ #1)}

% section font style
\sectionfont{\sffamily\Large}
\subsectionfont{\sffamily\normalsize}
\subsubsectionfont{\bf}

% line spreading and break
\hyphenpenalty=5000
\tolerance=20
\setlength{\parindent}{0em}
\setlength\parskip{0.5em}
\allowdisplaybreaks
\linespread{0.9}

% enumerate settings
% no break before enumerate
\setlist[enumerate]{itemsep=2pt,topsep=2pt}

% theorem
% latex theorem
% definition style
\theoremstyle{definition}
\newtheorem{theorem}{Theorem}[subsection]
\newtheorem{axiom}{Axiom}[section]
\newtheorem{definition}{Definition}[section]
\newtheorem{example}{Example}[section]
\newtheorem{question}{Question}[section]
\newtheorem{exercise}{Exercise}[section]
\newtheorem*{exercise*}{Exercise}
\newtheorem{lemma}{Lemma}[section]
\newtheorem{proposition}{Proposition}[section]
\newtheorem{corollary}{Corollary}[section]
\newtheorem*{theorem*}{Theorem}
\newtheorem{problem}{Problem}
% remark style
\theoremstyle{remark}
\newtheorem*{remark}{Remark}
\newtheorem*{solution}{Solution}
\newtheorem*{claim}{Claim}


% paragraph indent
\setlength{\parindent}{0em}
\setlength\parskip{0.5em}

\newcommand\Code{PHY5410 FA22}
\newcommand\Ass{HW13}
\newcommand\name{Haoran Sun}
\newcommand\mail{haoransun@link.cuhk.edu.cn}

\title{{\lms \Code \ \Ass}}
\author{\lms \name \ (\href{mailto:\mail}{\mail})}
\date{\sffamily \today}

\makeatletter
% \let\Title\@title
\let\theauthor\@author
\let\thedate\@date

\fancypagestyle{plain}{%
    \fancyhf{}
    \lhead{\sffamily \Ass}
    \rhead{\sffamily \name}
    \rfoot{\sffamily\thepage}

    % # 页脚自定义
    \fancyfoot[L]{
        \begin{minipage}[c]{0.06\textwidth}
            \includegraphics[height=7.5mm]{logo2.png}
        \end{minipage}
    }
}
\fancypagestyle{title}{%
    \fancyhf{}
    \renewcommand{\headrulewidth}{0pt}
    % \lhead{\Title}
    % \rhead{\theauthor}
    \rfoot{\sffamily\thepage}

    % # 页脚自定义
    \fancyfoot[L]{
        \begin{minipage}[c]{0.06\textwidth}
            \includegraphics[height=7.5mm]{logo2.png}
        \end{minipage}
    }
}
\fancyfootoffset[L]{0.3cm}

% re-define title format
\makeatletter
\renewcommand{\maketitle}{\bgroup\setlength{\parindent}{0pt}
\begin{flushleft}
  \textbf{\Large\@title}

  \@author
\end{flushleft}\egroup
}
\makeatother

\pagestyle{plain}

% lstlisting settings
\lstset{
    basicstyle=\linespread{0.7}\footnotesize,
    breaklines=true,
    basewidth=0.5em
}


\begin{document}
\maketitle
\thispagestyle{title}
% \begin{multicols*}{2}

% \begin{remark}
%     $V_\epsilon(x)$ is used to denote a $\epsilon$-neighborhood
%     \begin{align*}
%         V_\epsilon(x) = B_\epsilon(x)\setminus\{x\}
%     \end{align*}
% \end{remark}

\begin{problem}[5.1]\
\begin{enumerate}[label=(\alph*)]
\item Since $\sigma^i\sigma^j = \epsilon_{ijk}\sigma^k$ ($i\neq j$) and $(\sigma^i)^2=\mathbb{1}$,
then
\begin{align*}
    \alpha^i\alpha^j + \alpha^j\alpha^i &= 
    \begin{bmatrix}
        \sigma^i\sigma^j + \sigma^j\sigma^i & 0\\
        0 & \sigma^i\sigma^j + \sigma^j\sigma^i
    \end{bmatrix}
    = 2\delta_{ij}\mathbb{1}
\end{align*}

\item 
\begin{align*}
    \alpha^i\beta + \beta\alpha^i &= 
    \begin{bmatrix}
        0 & -\sigma^i\\
        \sigma^i & 0 
    \end{bmatrix} + 
    \begin{bmatrix}
        0 & \sigma^i\\
        -\sigma^i & 0 
    \end{bmatrix} = 0
\end{align*}

\item 
\begin{align*}
    (\alpha^i)^2 &= 
    \begin{bmatrix}
        \sigma^i\sigma^i & 0\\
        0 & \sigma^i\sigma^i
    \end{bmatrix} = \mathbb{1}\quad
    \beta^2 = 
    \begin{bmatrix}
        \mathbb{1}^2 & 0 \\
        0 & (-\mathbb{1})^2
    \end{bmatrix} = \mathbb{1}
\end{align*}

\end{enumerate}
\end{problem}


\begin{problem}[5.3]
Since
\begin{align*}
    i\hbar\partial_t\psi &= 
    \left[c\alpha^k\left(-i\hbar \partial_k - \frac{e}{c}A_k\right)+ \beta mc^2\right]\psi\\
    \Rightarrow
    E\psi &= 
    \left[c\alpha^k\left(-i\hbar \partial_k - \frac{e}{c}A_k\right)+ \beta mc^2\right]\psi\\
    E^2\psi &= 
    \left[c\alpha^k\left(-i\hbar \partial_k - \frac{e}{c}A_k\right)+ \beta mc^2\right]^2\psi\\
\end{align*}
the equation is summed over $k=1,2,3$.
Hence
\begin{align*}
    E^2\psi &= 
    \left[
    c\alpha^i\left(-i\hbar \partial_i - \frac{e}{c}A_i\right)
    c\alpha^j\left(-i\hbar \partial_j - \frac{e}{c}A_j\right)
    + (\alpha^k\beta + \beta\alpha^k)\cdots
    + \beta^2 m^2c^4
    \right]\psi\\
    &= \left[
    c^2\left(-i\hbar \partial_i - \frac{e}{c}A_i\right)^2
    + m^2c^4
    \right]\psi
\end{align*}
Give that $A_1=A_3=0$, $A_2=Bx$, 
using the conclusion from {\lms HW04}, where energy levels of such systems looks like
\begin{align*}
    E_n &= \left(n + \frac{1}{2}\right)\hbar\omega_B\quad
    \omega_B = \frac{eB}{mc}
\end{align*}
Hence in this system
\begin{align*}
    E_n^2 &= \left[
    (2n+1)\hbar\omega_B mc^2 + m^2c^4
    \right]\\
    \Rightarrow E_n &= 
    \sqrt{(2n+1)\hbar\omega_B mc^2 + m^2c^4}
\end{align*}
where $\omega_B = eB/mc$ (suppose $E\geq 0$).

\end{problem}


\begin{problem}[6.2]
Note that
\begin{align*}
    x^\mu &= g^{\mu\nu}x_\nu\quad
    x_\mu = g_{\mu\nu}x^\nu
\end{align*}
and 
\begin{align*}
    {x'}^\mu &= {\Lambda^\mu}_\nu x^\nu\quad
    x'_\mu = g_{\mu\nu}{\Lambda^\nu}_\sigma g^{\sigma\rho}x_\rho = 
    {\Lambda_\rho}^\mu x_\rho
\end{align*}
Note that if we sum over $b$
\begin{align*}
    {\Lambda^a}_b{\Lambda_c}^b &=
    {\Lambda^a}_b g_{b\nu}{\Lambda^\nu}_\sigma g^{\sigma c}
    = g_{a\sigma}g^{\sigma c}
    = \delta_{ac}
\end{align*}
Then $({\Lambda^\mu}_\nu)^{-1} = {\Lambda_\nu}^\mu$.
Hence
\begin{align*}
    \partial'_\mu &= 
    \frac{\partial x^\nu}{\partial {x'}^\mu}\partial_\nu = {\Lambda_\mu}^\nu\partial_\nu\\
    {\partial'}^\mu &= 
    \frac{\partial x_\nu}{\partial x'_\mu}\partial_\nu = {\Lambda^\nu}_\mu\partial^\nu
\end{align*}

\end{problem}


\begin{problem}[6.4]
The Dirac equation reads
\begin{align*}
    [-\gamma^\mu(i\hbar \partial_\mu - \frac{e}{c}A_\mu) + mc]\psi &= 0
\end{align*}
Hence
\begin{align*}
    [-\gamma^\nu(i\hbar \partial_\nu - \frac{e}{c}A_\nu) - mc]
    [-\gamma^\mu(i\hbar \partial_\mu - \frac{e}{c}A_\mu) + mc]
    \psi &= 0\\
    [
    \gamma^\nu
    \gamma^\mu
    (i\hbar \partial_\nu - \frac{e}{c}A_\nu)
    (i\hbar \partial_\mu - \frac{e}{c}A_\mu)
    - m^2c^2]
    \psi &= 0
\end{align*}
Since $\gamma^\nu\gamma^\mu = g^{\nu\mu}\mathbb{1} + i\sigma^{\mu\nu}$, 
where $\sigma^{\mu\nu}=i[\gamma^\mu,\gamma^\nu]/2$.
Then
\begin{align*}
    [\gamma^\nu\gamma^\mu
    (i\hbar \partial_\nu - \frac{e}{c}A_\nu)
    (i\hbar \partial_\mu - \frac{e}{c}A_\mu)
    - m^2c^2]
    \psi &= 0\\
    [(g^{\nu\mu}\mathbb{1} + i\sigma^{\mu\nu})
    (i\hbar \partial_\nu - \frac{e}{c}A_\nu)
    (i\hbar \partial_\mu - \frac{e}{c}A_\mu)
    - m^2c^2]
    \psi &= 0\\
    [
    (\partial - \frac{e}{c}A)^2
    + i\sigma^{\mu\nu}
    (i\hbar \partial_\nu - \frac{e}{c}A_\nu)
    (i\hbar \partial_\mu - \frac{e}{c}A_\mu)
    - m^2c^2]
    \psi &= 0\\
    [
    (\partial - \frac{e}{c}A)^2
    - i\sigma^{\mu\nu}
    (i\hbar \partial_\mu - \frac{e}{c}A_\mu)
    (i\hbar \partial_\nu - \frac{e}{c}A_\nu)
    - m^2c^2]
    \psi &= 0
\end{align*}
Note that
\begin{align*}
    i\sigma^{\mu\nu}
    (i\hbar \partial_\mu - \frac{e}{c}A_\mu)
    (i\hbar \partial_\nu - \frac{e}{c}A_\nu)
    &= \frac{1}{2}[
    i\sigma^{\mu\nu}
    (i\hbar \partial_\mu - \frac{e}{c}A_\mu)
    (i\hbar \partial_\nu - \frac{e}{c}A_\nu)
    + 
    i\sigma^{\nu\mu}
    (i\hbar \partial_\nu - \frac{e}{c}A_\nu)
    (i\hbar \partial_\mu - \frac{e}{c}A_\mu)
    ]\\
    &= \frac{1}{2}[
    i\sigma^{\mu\nu}
    (i\hbar \partial_\mu - \frac{e}{c}A_\mu)
    (i\hbar \partial_\nu - \frac{e}{c}A_\nu)
    -
    i\sigma^{\mu\nu}
    (i\hbar \partial_\nu - \frac{e}{c}A_\nu)
    (i\hbar \partial_\mu - \frac{e}{c}A_\mu)
    ]\\
    &= \frac{i}{2}\sigma^{\mu\nu}[
    (i\hbar \partial_\mu - \frac{e}{c}A_\mu),
    (i\hbar \partial_\nu - \frac{e}{c}A_\nu)
    ]\\
    &= \frac{\hbar}{2}\frac{e}{c}\sigma^{\mu\nu}F_{\mu\nu} 
    \quad\text{\color{red}I don't understand how this step works}
\end{align*}
Therefore 
\begin{align*}
    [(\partial - \frac{e}{c}A)^2
    -\frac{he}{2c}\sigma^{\mu\nu}F_{\mu\nu}
    - m^2c^2]
    \psi &= 0
\end{align*}
Note that 
\begin{align*}
    A &= \begin{bmatrix}
        \Phi\\ A_1 \\ A_2 \\ A_3
    \end{bmatrix}\quad
    E = \nabla\Phi = \begin{bmatrix}
        E_1 \\ E_2 \\ E_3
    \end{bmatrix}\quad
    B = \nabla\times
    \begin{bmatrix}
        A_1 \\ A_2 \\ A_3
    \end{bmatrix} = \begin{bmatrix}
        B_1 \\ B_2 \\ B_3
    \end{bmatrix}
\end{align*}
Therefore
\begin{align*}
    F &= \begin{bmatrix}
        0 & -E_1 & -E_2 & -E_3\\
        E_1 & 0 & B_3 & -B_2\\
        E_2 & -B_3 & 0 & B_1\\
        E_3 & B_2 & -B_1 & 0
    \end{bmatrix}
\end{align*}
Note that
\begin{align*}
    \sigma^{0\nu} &= 
    \frac{i}{2}[\gamma^0, \gamma^\nu] = \frac{i}{2}[\beta,\beta\alpha^\nu]
    = \frac{i}{2}\beta[\beta, \alpha^\nu] = \frac{i}{2}\beta(2\beta\alpha^\nu)
    = i\alpha^\nu
\end{align*}
Hence for $\mu=0$ or $\nu=0$, we have 
\begin{align*}
    \sigma^{\mu\nu}F_{\mu\nu} &= 2\sigma^{0\nu}F_{0\nu}
    = -2i\alpha\cdot E
\end{align*}
For all $\mu,\nu\geq 1$, we have
\begin{align*}
    \sigma^{\mu\nu} &=
    \frac{i}{2}[\gamma^\nu, \gamma^\nu]
    = \frac{i}{2}[\beta\alpha^\mu,\beta\alpha^\nu]
    = \frac{i}{2}[\alpha^\nu, \alpha^\mu]
    = -\frac{i}{2}2i\epsilon_{\nu\mu\rho}\Sigma_\rho
    = \epsilon_{\mu\nu\rho}\Sigma_\rho
\end{align*}
Then for all $\mu,\nu\geq 1$
\begin{align*}
    \sigma^{\mu\nu}F_{\mu\nu} &= 
    \sigma_{\mu\nu\rho}\Sigma_\rho \epsilon_{\mu\nu\sigma}B_\sigma
    = 2\delta_{\rho\sigma}\Sigma_\rho B_\sigma
    = 2\Sigma\cdot B
\end{align*}
Hence 
\begin{align*}
    \sigma^{\mu\nu}F_{\mu\nu} &= 
    -2i\alpha\cdot E + 2\Sigma\cdot B
    = -2i (\alpha\cdot E + i\Sigma\cdot B)
\end{align*}
Then the Dirac equation becomes
\begin{align*}
    [(\partial - \frac{e}{c}A)^2
    +\frac{i\hbar e}{c}(\alpha\cdot E + i\Sigma\cdot B)
    - m^2c^2]
    \psi &= 0
\end{align*}

\end{problem}


% \end{multicols*}
\end{document}

