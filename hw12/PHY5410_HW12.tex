\documentclass[twoside,11pt]{article}
\usepackage[left=1in, right=1in, top=1in, bottom=1in]{geometry}
\usepackage{amsmath}
\usepackage{amssymb}
\usepackage{amsfonts}
\usepackage{mathtools}
\usepackage{amsthm}
\usepackage{fancyhdr}
\usepackage{enumitem}
\usepackage{siunitx}
\usepackage{booktabs}
\usepackage[hidelinks]{hyperref}
\usepackage{sectsty}
\usepackage{mathrsfs} % mathscr
\usepackage{tikz}
\usepackage{pgfplots}
\usepackage{multicol}
\usepackage{listings}
% \usepackage{amsart}
\usepackage{fontspec}
\usepackage{soul}


% allow H option of figure
\usepackage{float}

% math font (libertine)
\usepackage{libertinus-otf}

% braket
\usepackage{braket}

% physics
% \usepackage{physics}

% define latin modern font environment
\newcommand{\lms}{\fontfamily{lmss}\selectfont} % Latin Modern Roman
% \newcommand{\lmss}{\fontfamily{lmss}\selectfont} % Latin Modern Sans
% \newcommand{\lmss}{\fontfamily{lmtt}\selectfont} % Latin Modern Mono

% % change mathcal shape
% \usepackage[mathcal]{eucal}


% define math operators
\newcommand{\FF}{\mathbb{F}}
\newcommand{\RR}{\mathbb{R}}
\newcommand{\NN}{\mathbb{N}}
\newcommand{\ZZ}{\mathbb{Z}}
\newcommand{\QQ}{\mathbb{Q}}
\newcommand{\XX}{\mathbb{Y}}
\newcommand{\CL}{\mathcal{L}}
% \renewcommand{\d}{\mathrm{d}}
\renewcommand*\d{\mathop{}\!\mathrm{d}}
\DeclareMathOperator*{\argmax}{arg\,max}
\DeclareMathOperator*{\argmin}{arg\,min}
\DeclareMathOperator{\im}{im}
\DeclareMathOperator{\id}{id}
\DeclareMathOperator{\erf}{erf}
\renewcommand{\mod}[1]{\ (\mathrm{mod}\ #1)}

% section font style
\sectionfont{\sffamily\Large}
\subsectionfont{\sffamily\normalsize}
\subsubsectionfont{\bf}

% line spreading and break
\hyphenpenalty=5000
\tolerance=20
\setlength{\parindent}{0em}
\setlength\parskip{0.5em}
\allowdisplaybreaks
\linespread{0.9}

% enumerate settings
% no break before enumerate
\setlist[enumerate]{itemsep=2pt,topsep=2pt}

% theorem
% latex theorem
% definition style
\theoremstyle{definition}
\newtheorem{theorem}{Theorem}[subsection]
\newtheorem{axiom}{Axiom}[section]
\newtheorem{definition}{Definition}[section]
\newtheorem{example}{Example}[section]
\newtheorem{question}{Question}[section]
\newtheorem{exercise}{Exercise}[section]
\newtheorem*{exercise*}{Exercise}
\newtheorem{lemma}{Lemma}[section]
\newtheorem{proposition}{Proposition}[section]
\newtheorem{corollary}{Corollary}[section]
\newtheorem*{theorem*}{Theorem}
\newtheorem{problem}{Problem}
% remark style
\theoremstyle{remark}
\newtheorem*{remark}{Remark}
\newtheorem*{solution}{Solution}
\newtheorem*{claim}{Claim}


% paragraph indent
\setlength{\parindent}{0em}
\setlength\parskip{0.5em}

\newcommand\Code{PHY5410 FA22}
\newcommand\Ass{HW12}
\newcommand\name{Haoran Sun}
\newcommand\mail{haoransun@link.cuhk.edu.cn}

\title{{\lms \Code \ \Ass}}
\author{\lms \name \ (\href{mailto:\mail}{\mail})}
\date{\sffamily \today}

\makeatletter
% \let\Title\@title
\let\theauthor\@author
\let\thedate\@date

\fancypagestyle{plain}{%
    \fancyhf{}
    \lhead{\sffamily \Ass}
    \rhead{\sffamily \name}
    \rfoot{\sffamily\thepage}

    % # 页脚自定义
    \fancyfoot[L]{
        \begin{minipage}[c]{0.06\textwidth}
            \includegraphics[height=7.5mm]{logo2.png}
        \end{minipage}
    }
}
\fancypagestyle{title}{%
    \fancyhf{}
    \renewcommand{\headrulewidth}{0pt}
    % \lhead{\Title}
    % \rhead{\theauthor}
    \rfoot{\sffamily\thepage}

    % # 页脚自定义
    \fancyfoot[L]{
        \begin{minipage}[c]{0.06\textwidth}
            \includegraphics[height=7.5mm]{logo2.png}
        \end{minipage}
    }
}
\fancyfootoffset[L]{0.3cm}

% re-define title format
\makeatletter
\renewcommand{\maketitle}{\bgroup\setlength{\parindent}{0pt}
\begin{flushleft}
  \textbf{\Large\@title}

  \@author
\end{flushleft}\egroup
}
\makeatother

\pagestyle{plain}

% lstlisting settings
\lstset{
    basicstyle=\linespread{0.7}\footnotesize,
    breaklines=true,
    basewidth=0.5em
}


\begin{document}
\maketitle
\thispagestyle{title}
% \begin{multicols*}{2}

% \begin{remark}
%     $V_\epsilon(x)$ is used to denote a $\epsilon$-neighborhood
%     \begin{align*}
%         V_\epsilon(x) = B_\epsilon(x)\setminus\{x\}
%     \end{align*}
% \end{remark}

\begin{problem}[1.10]\
\begin{enumerate}[label=(\alph*)]
\item Let $\ket{\phi}=\psi(\mathbf{x}')\ket{0}$, then we have
\begin{align*}
    n(\mathbf{x})\ket{\phi} &= 
    \psi^\dagger(\mathbf{x})\psi(\mathbf{x})
    \psi^\dagger(\mathbf{x}')\ket{0}
    = \psi^\dagger(\mathbf{x})[
        \psi^\dagger(\mathbf{x}')\psi(\mathbf{x})
        + \delta(\mathbf{x}-\mathbf{x}')
    ]\ket{0}\\
    &= \psi^\dagger(\mathbf{x})\psi^\dagger(\mathbf{x}')\psi(\mathbf{x})\ket{0}
    + \psi^\dagger(\mathbf{x})\delta(\mathbf{x}-\mathbf{x}')\ket{0}\\
    &= \psi^\dagger(\mathbf{x})\delta(\mathbf{x}-\mathbf{x}')\ket{0}
\end{align*}

\item Note that
\begin{align*}
    \hat{N} &= 
    \int\d\mathbf{x}\ n(\mathbf{x})
    = \int\d\mathbf{x}\ \sum_{ij} a_i^\dagger a_j\braket{i|\mathbf{x}}\braket{\mathbf{x}|j}
    = \sum_{ij} a_i^\dagger a_j\int\d\mathbf{x}\braket{i|\mathbf{x}}\braket{\mathbf{x}|j}
    = \sum_{ij} a_i^\dagger a_j\delta_{ij}
    = \sum_i a_i^\dagger a_j
\end{align*}
Using the properties where $[a_i, a_j]=0$, $[a_i, a_j^\dagger]=\delta_{ij}$,
then we have
\begin{align*}
    \psi(\mathbf{x})\hat{N} &= 
    \sum_{ij}a_ia_j^\dagger a_j\braket{\mathbf{x}|j}
    = \sum_{ij}(a_j^\dagger a_i + \delta_{ij})a_j\braket{\mathbf{x}|j}
    = \sum_{ij}a_j^\dagger a_ja_i\braket{\mathbf{x}|j}
    + \sum_{j} a_j\braket{\mathbf{x}|j}
    = \hat{N}\psi(\mathbf{x}) + \psi(\mathbf{x})
\end{align*}

\end{enumerate}
\end{problem}


\begin{problem}[2.1]\
\begin{enumerate}[label=(\alph*)]
\item Suppose $\mathbf{q}=0$, then
\begin{align*}
    S^0(0) &= 
    \frac{1}{N}
    \sum_{\mathbf{k}\sigma\mathbf{k}'\sigma'}
    \braket{\phi_0|
    a_{\mathbf{k}\sigma}^\dagger a_{\mathbf{k}\sigma}
    a_{\mathbf{k}'\sigma'}^\dagger a_{\mathbf{k}'\sigma'}
    |\phi_0}
    = 
    \frac{1}{N}
    \sum_{\mathbf{k}\sigma\mathbf{k}'\sigma'}
    \braket{\phi_0|
    n_{\mathbf{k},\sigma}n_{\mathbf{k}',\sigma'}
    |\phi_0}
    =
    \frac{1}{N}
    \sum_{\mathbf{k}\sigma\mathbf{k}'\sigma'}
    = N
\end{align*}

\item Suppose $\mathbf{q}\neq 0$, then
\begin{align*}
    S^0(\mathbf{q}) &=
    \frac{1}{N}
    \sum_{\mathbf{k}\sigma\mathbf{k}'\sigma'}
    \braket{\phi_0|
    a_{\mathbf{k}\sigma}^\dagger a_{\mathbf{k}+\mathbf{q}\sigma}
    a_{\mathbf{k}'\sigma'}^\dagger a_{\mathbf{k}'-\mathbf{q}\sigma'}
    |\phi_0}
\end{align*}
Since $\ket{\phi_0}$ is the ground state, we should have $|\mathbf{k}'-\mathbf{q}|<k_F$.
Also, since $\mathbf{k}'\neq\mathbf{k}'-\mathbf{q}$, we should have
$|\mathbf{k}'|\geq k_F$.
Therefore, $a_{\mathbf{k}'\sigma'}^\dagger a_{\mathbf{k}'-\mathbf{q}\sigma'}\ket{\phi_0}$
is an excited state, which pops a ground state $\mathbf{k}'-\mathbf{q},\sigma'$ out
and add an fermion on the high-energy state $\mathbf{k}',\sigma'$.
Denote this state as
\begin{align*}
    a^\dagger_{\mathbf{k}'\sigma'}
    a_{\mathbf{k}'-\mathbf{q}\sigma'}\ket{\phi_0} &=
    \ket{\phi_{\mathbf{k'}-\mathbf{q}\sigma'}^{\mathbf{k}'\sigma'}}
\end{align*}
Similarly, we have
\begin{align*}
    \bra{\phi_0}a_{\mathbf{k}\sigma}^\dagger
    a_{\mathbf{k}+\mathbf{q}\sigma}
    &= 
    \bra{\phi_{\mathbf{k}\sigma}^{\mathbf{k}+\mathbf{q}\sigma}}
\end{align*}
where $|\mathbf{k}|<k_F$ and $|\mathbf{k}+\mathbf{q}|>k_F$.
Then we have
\begin{align*}
    S^0(\mathbf{q}) &= 
    \frac{1}{N}
    \sum_{\mathbf{k}\sigma\mathbf{k}'\sigma'}
    \braket{\phi_{\mathbf{k}\sigma}^{\mathbf{k}+\mathbf{q}\sigma}|
    \phi_{\mathbf{k'}-\mathbf{q}\sigma'}^{\mathbf{k}'\sigma'}}
    =
    \frac{1}{N}
    \sum_{\mathbf{k}\sigma\mathbf{k}'\sigma'}
    \delta_{\sigma\sigma'}\delta_{\mathbf{k},\mathbf{k}'-\mathbf{q}}
    \delta_{\mathbf{k}',\mathbf{k}+\mathbf{q}}
    =
    \frac{1}{N}
    \sum_{\mathbf{k}\sigma} 1
\end{align*}
note that the sum is taken with the restraint
$\mathbf{k}\in\Omega = \{\mathbf{k}||\mathbf{k}|<k_F, |\mathbf{k}+\mathbf{q}|>k_F\}$.
Using the continuum limit we have
\begin{align*}
    S^0(\mathbf{q}) &= 
    2\frac{V}{N}\int_\Omega \frac{\d\mathbf{k}}{(2\pi)^3}
    = \begin{cases}
        \frac{2+3|\mathbf{q}/2k_F|-|\mathbf{q}/2k_F|^3}{4} & 0 < |\mathbf{q}| \leq 2k_F\\
        1 & |\mathbf{q}| > 2k_F
    \end{cases}
\end{align*}

\end{enumerate}
\end{problem}


\begin{problem}[2.6]\
\begin{enumerate}[label=(\alph*)]
\item 
\begin{align*}
    e^{-\alpha a^\dagger} a e^{\alpha a^\dagger}
    &= (1 - \alpha a^\dagger) a (1 + \alpha a^\dagger)
    = (a - \alpha a^\dagger a)(1 + \alpha a^\dagger)
    = a - \alpha^2 a^\dagger + \alpha(aa^\dagger - a^\dagger a)\\
    e^{-\alpha a} a^\dagger e^{\alpha a}
    &= (1 - \alpha a) a^\dagger (1 + \alpha a)
    = (a^\dagger - \alpha aa^\dagger) (1+\alpha a)
    = a^\dagger - \alpha^2 a + \alpha(a^\dagger a - aa^\dagger)
\end{align*}

\item 
\begin{align*}
    e^{\alpha a^\dagger a}a e^{-\alpha a^\dagger a}
    &= 
    (1 + \alpha a^\dagger a + \cdots) a \sum_n \frac{(-\alpha)^n}{n!}(a^\dagger a)^n\\
    &= (1 + \alpha a^\dagger a + \cdots) \sum_n\frac{(-\alpha)^n}{n!}a(a^\dagger a)(a^\dagger a)^{n-1}\\
    &= (1 + \alpha a^\dagger a + \cdots) \sum_n\frac{(-\alpha)^n}{n!}(1-a^\dagger a)a(a^\dagger a)^{n-1}\\
    &= (1 + \alpha a^\dagger a + \cdots) \sum_n\frac{(-\alpha)^n}{n!}a(a^\dagger a)^{n-1}\\
    &= (1 + \alpha a^\dagger a + \cdots) \sum_n\frac{(-\alpha)^n}{n!}a = e^{-\alpha} a\\
    e^{\alpha a^\dagger}a^\dagger e^{-\alpha a^\dagger a}
    &= \sum_n\frac{\alpha^n}{n!}(a^\dagger a)^n a^\dagger (1 + \alpha a^\dagger a + \cdots)\\
    &= \sum_n\frac{\alpha^n}{n!}(a^\dagger a)^{n-1}(a^\dagger a)a^\dagger (1 + \alpha a^\dagger a + \cdots)\\
    &= \sum_n\frac{\alpha^n}{n!}(a^\dagger a)^{n-1} a^\dagger(1 + \alpha a^\dagger a + \cdots)\\
    &= \sum_n\frac{\alpha^n}{n!}a^\dagger (1 + \alpha a^\dagger a + \cdots)\\
    &= e^{\alpha}a^\dagger
\end{align*}

\end{enumerate}
\end{problem}



% \end{multicols*}
\end{document}

